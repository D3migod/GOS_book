\chapter{Формула Остроградского-Гаусса. Соленоидальные векторные поля.}
\section{Формула Остроградского-Гаусса}

\begin{defn}
Область $G \subset \bbR^3$ называется элементарной относительно оси $z$ в прямоугольной декартовой системе координат $(O, x, y, z)$, если 

\begin{equation} \label{ch15eq1}
G = \{ (x,y,z) \in \bbR^3: \: (x,y) \in g, \: \phi_1(x,y) < z < \phi_2(x,y) \: \forall (x,y) \in g\}, 
\end{equation}

где $g$ - ограниченная область в $\bbR^2$ с кусочно-гладкой границей $\partial g$ и функции $\phi_1, \phi_2$ определены, непрерывны на замыкании области $\overline{g}$ и непрерывно-дифференцируемы на $g$.
\end{defn}

Аналогично определяется область в $\bbR^3$, элементарная относительно осей $x,y$.

\begin{notion}
Так как область $g$ ограничена и ее граница является кусочно-гладкой, то граница имеет нулевую меру Жордана, а $g$ - измеримо по Жордану, $\partial g$ - компакт.
\end{notion}

\begin{lemm} \label{ch15lemm1}
Пусть область $G$ элементарна относительно оси $z(x,y)$. Тогда если функция $f$ определена и непрерывна на замыкании $\overline{G}$, и ее частная производная $\frac{\partial f}{\partial z}, \textit{или }\frac{\partial f}{\partial x}, \textit{или }\frac{\partial f}{\partial y}$ непрерывна и ограничена на $G$, то 

\begin{equation} \label{ch15eq2z}
\iiint\limits_{G} \frac{\partial f}{\partial z}\,dx\,dy\,dz = \iint\limits_{\partial	G_{\textit{внеш}}} f \,dx\,dy,
\end{equation}

или

\begin{equation} \label{ch15eq2x}
\iiint\limits_{G} \frac{\partial f}{\partial x}\,dx\,dy\,dz = \iint\limits_{\partial	G_{\textit{внеш}}} f \,dy\,dz,
\end{equation}

или

\begin{equation} \label{ch15eq2y}
\iiint\limits_{G} \frac{\partial f}{\partial y}\,dx\,dy\,dz = \iint\limits_{\partial	G_{\textit{внеш}}} f \,dx\,dz,
\end{equation}

где поверхностные интегралы берутся по внешней относительно $G$ стороне повехности $\partial G$.
\end{lemm}

\begin{proof}
Область $G$ задана условием $(\ref{ch15eq1}) \; \Rightarrow \; \partial G = S_1 \cup S_2 \cup S_0$, 
\addpicture{pictures/ch15pict1}{0.25}
где $S_{1,2} = \{ (x,y) \in \overline{g}, \; z = \phi_{1,2}(x,y) \}$, $S_0 = \{ (x,y) \in \partial g, \phi_1(x,y) \le z \le \phi_2(x,y) \}$ 

$G = \{ (x,y,z) \in \bbR^3: \; (x,y) \in g, \; \phi_1(x,y) < z < \phi_2(x,y) \; \forall (x,y) \in g \} \; \Rightarrow \; \partial G$ -- кусочно-гладкая поверхнотсь.

$\displaystyle\iint\limits_{S_{2_{\textit{внеш}}}} f \,dx\,dy = \displaystyle\iint\limits_g f(x,y,\phi_2(x,y)) \,dx\,dy$

$\displaystyle\iint\limits_{S_{1_{\textit{внеш}}}} f \,dx\,dy = - \displaystyle\iint\limits_g f(x,y,\phi_1(x,y)) \,dx\,dy$

$\displaystyle\iint\limits_{S_{0_{\textit{внеш}}}} f \,dx\,dy = 0$, так как на $S_0 \; \overset{\to}{n}_{\textit{внеш}(x,y,z) = (n_x, n_y, 0)}$

$\Rightarrow$

$\displaystyle\iint\limits_G f \,dx\,dy = \displaystyle\iint\limits_g (f(x,y,\phi_2(x,y)) - f(x,y,\phi_1(x,y))) \,dx \,dy$

$\displaystyle\iiint\limits_G \dfrac{\partial f}{\partial z} \,dx \,dy \,dz = \displaystyle\iint\limits_g \,dx \,dy \displaystyle\int\limits_{\phi_1(x,y)}^{\phi_2(x,y)} \dfrac{\partial f}{\partial z}(x,y,z) \,dz = \displaystyle\iint\limits_g f(x,y,\phi_2) - f(x,y,\phi_1) \,dx \,dy$

\end{proof}

\begin{cons}
Лемма $\ref{ch15lemm1}$ для $x$ и для $y$ доказывается аналогично.
\end{cons}

\begin{thm} [Теорема Остроградского-Гаусса]
Пусть область $G \subset \bbR^3$ элементарна относительно всех координатных осей. Тогда если функции $P,Q,R$ определены и непрерывны на $\overline{G}$, а частные производные $P'_x, Q'_y, R'_z$ - непрерывны и ограничены на $G$, то справедлива формула Остроградского-Гаусса:

\begin{equation} \label{ch15eq3}
\iiint\limits_G (P'_x + Q'_y + R'_z) \,dx\,dy\,dz = \iint\limits_{\partial G_{\textit{внеш}}} P\,dy\,dz + Q\,dz\,dx + R\,dx\,dy,
\end{equation}
где $\partial G_{\textit{внеш}}$ -- внешнаяя относительно области $G$ сторона поверхности $G$.

\end{thm}
\begin{proof}
$(\ref{ch15eq3}) \Leftarrow (\ref{ch15eq2x}), (\ref{ch15eq2y}), (\ref{ch15eq2z})$
\end{proof}

\begin{lemm}
Пусть область $G \subset \bbR^3$ можно разрезать кусочно-гладкой поверхностью $\Sigma \subset G$ на две области $G_1$ и $G_2$, каждая из которых элементарна относительно координатной оси $z$. Тогда если функция $f$ определена и непрерывна на $\overline{G}$ и ее частная производная $f'_z$ непрерывна и ограничена на $G$, то справедлива формула ($\ref{ch15eq2z}$).
\end{lemm}

\begin{proof}
$G = G_1 \cup G_2$

$\partial G_{1_{\textit{внеш}}} = \partial G'_{\textit{внеш}} \cup \Sigma_{1_{\textit{внеш}}}$

$\partial G_{2_{\textit{внеш}}} = \partial G''_{\textit{внеш}} \cup \Sigma_{2_{\textit{внеш}}}$

$\partial G_{\textit{внеш}} = \partial G'_{\textit{внеш}} \cup \partial G''_{\textit{внеш}}$

\begin{equation*}
\left.\begin{aligned}
\Sigma_{1_{\textit{внеш}}} \\ 
 \Sigma_{2_{\textit{внеш}}}
\end{aligned} \right\} \quad  \textit{это поверхности } \Sigma \textit{с противоположными ориентациями}.
\end{equation*}
\addpicture{pictures/ch15pict2}{0.12}
\begin{equation} \label{ch15eqA}
\iint\limits_{\partial G_{1_{\textit{внеш}}}} f \,dx\,dy + \iint\limits_{\partial G_{2_{\textit{внеш}}}} f \,dx\,dy = \iint\limits_{G_{\textit{внеш}}} f\,dx\,dy
\end{equation}

\begin{equation} \label{ch15eqB}
\iiint\limits_{G_{1,2}} \frac{\partial f}{\partial z}\,dx\,dy\,dz = \iint\limits_{\partial G_{1,2_{\textit{внеш}}}} f \,dx,dy
\end{equation}

\begin{equation} \label{ch15eqC}
\iiint\limits_{G} \frac{\partial f}{\partial z}\,dx\,dy\,dz = \iiint\limits_{G_1} \frac{\partial f}{\partial z}\,dx\,dy\,dz + \iiint\limits_{G_2} \frac{\partial f}{\partial z}\,dx\,dy\,dz 
\end{equation}

$(\ref{ch15eqA}, \ref{ch15eqB}, \ref{ch15eqC}) \; \Rightarrow \; (\ref{ch15eq2z})$
\end{proof}

\begin{thm}
Пусть область $G \subset \bbR^3$ такая, что для каждой их трех координатных осей ее можно конечным числом кусочно-гладких поверхностей разрезать на конечное число областей, элементарных относительно соответсвующей оси. Тогда если функции $P,Q,R$ непрерывны на $\overline{G}$, а их частные производные $P'_x, Q'_y, R'_z$ непрерывны и ограничены на $G$, то справедлива формула Остроградского-Гаусса ($\ref{ch15eq3}$).
\end{thm}

\begin{thm}
Пусть область $G$ -- ограниченная область в $\bbR^3$, граница которой состоит из конечного числа гладких поверхностей. Пусть функции $P,Q,R$ определены и непрерывны на замыкании $\overline{G}$. Тогда если она непрерывно-дифференцируема на $G$, то справедлива формула ($\ref{ch15eq3}$).
\end{thm}
\section{Соленоидальные векторные поля.}

\begin{defn}
Непрерывно-дифференцируемое векторное поле $\overset{\to}{a}(M), \; M \in G$ называется соленоидальным в области $G$, если 
\begin{equation} \label{ch15.2eq1}
\forall M \in G \quad \Div \: \overset{\to}{a}(M) = 0
\end{equation}
\end{defn}

\begin{thm}
Непрерывно-дифференцируемое векторное поле $\overset{\to}{a}(M), \; M \in G$ является соленоидальным в области $G$ $\Leftrightarrow$ для любой замкнутой области $\overline{g}$ такой, что $\overline{g} \subset G$ и $\partial g$ - кусочно-гладкая граница области $g$, выполняется равенство:

\begin{equation} \label{ch15.2eq2}
\iint\limits_{\partial g_{\textit{внеш}}} \overset{\to}{a} \,d\overset{\to}{s} = 0
\end{equation}
\end{thm}

\begin{proof}
Пусть выполняется ($\ref{ch15.2eq1}$). Пусть $g$ -- область: $\overline{g} \subset G$ и $\partial g$ - кусочно-гладкая поверхность.

Тогда так как $\overset{\to}{a}$ -- непрерывно-дифференциуемое векторное поле в $G$, по формуле Остроградского-Гаусса:

$$
\iint\limits_{\partial \overline{g}} \overset{\to}{a} \,d\overset{\to}{s} = \iiint\limits_g \Div \: \overset{\to}{a}(M) \,dg = 0, \; \textit{из } (\ref{ch15.2eq1}) 
$$

$\Rightarrow$ из ($\ref{ch15.2eq1}$) $\Rightarrow$ ($\ref{ch15.2eq2}$)

Пусть выполняется ($\ref{ch15.2eq2}$), $M \in G$.
$\Rightarrow \; \exists \delta > 0: \; O_\delta \subset G$

$\Rightarrow \; \Div \: \overset{\to}{a}(M) = \lim\limits_{r \to +0}\left( \frac{1}{\frac{4}{3}\pi r^3}\iint\limits_{S^{+}_M(r)}a \,d\overset{\to}{s} \right) = 0,$ где $S^{+}_M(r)$ -- сфера с центром в точке $M$ радиуса $r, \: r \in (0, \delta)$.

$(\ref{ch15.2eq2}) \Rightarrow (\ref{ch15.2eq1})$
\end{proof}

\begin{defn}
Непрерывно-дифференцируемая векторная фукнция $\overset{\to}{b}(M), \; M \in G$, называется векторным потенциалом векторного поля $\overset{\to}{a}(M), \; M \in G \subset \bbR^3$, в области $G$, если 
$$
\forall M \in G: \; \overset{\to}{a}(M) = \rot \: \overset{\to}{b}(M)
$$
\end{defn}

\begin{lemm}
Если непрерывно-дифференцируемое векторное поле $\overset{\to}{a}(M), \; M \in G$, имеет на области $G$ векторный потенциал, то оно соленоидально на $G$. 
\end{lemm}

\begin{proof}
$\overset{\to}{a}(M)$ -- непрерывно-дифференцируемо по условию.
Пусть $\forall M \in G \quad \overset{\to}{a}(M) = \rot \: \overset{\to}{b}(M)$.
$\Rightarrow \quad \exists \Div \: \overset{\to}{a}(M) = \Div(\rot \: \overset{\to}{b}(M)) = 0$ в $G$.
\end{proof}

\begin{defn}
Замкнутой кусочно-гладкой поверхностью называется ограниченная кусочно-гладкая поверхность, не имеющая края. \addpicture{pictures/ch15pict3}{0.25} При этом, если $S = \sum\limits_{j = 1}^m S_j, \quad \textit{где } \forall j \in \overline{1,m}, \quad S_j$ -- простая гладкая поверхность с кусочно-гладким краем $\partial S_j$, то каждый гладкий участок края $\partial S_j$ совпадает с гладким участком края $\partial S_i$, но их ориентации
 противоположны. Суммарный край $\partial S = \sum\limits_{j = 1}^m \partial S_j$ -- пуст.
\end{defn}

\begin{thm}
Если непрерывно-дифференцируемое векторное поле $\overset{\to}{a}(M), \; M \in G$, имеет в области $G$ векторный потенциал $\overset{\to}{b}$, то для любой замкнутой кусочно-гладкой поверхности $S: \; S\subset G$ выполняется равенство: $\iint\limits_S \overset{\to}{a}\,d\overset{\to}{s} = 0$. (Область $\Omega$, ограниченная поверхностью $S$, может не лежать в $G$, хотя $S \subset G$.)
\end{thm}

\begin{proof}
Пусть $S_1$ -- гладкий кусок кусочно-гладкой поверхности $S$ без края. Пусть $\gamma_1 = \partial S_1, \; \gamma_1$ -- край $S_1$. Тогда поверхность $S_2 = S \setminus S_1$ имеет край $\gamma_2 = \partial S_2 = (\gamma_1)^{-1}$.

$\overset{\to}{a}$ -- непрерывно-дифференкируемая вектор функция на области $G$.

$S_2, S_1 \subset G$.

$\Rightarrow$ на $S_2, S_1$ справедлива формула Стокса.

$\iint\limits_S \overset{\to}{a}\,d\overset{\to}{s} = \iint\limits_{S_1} \overset{\to}{a}\,d\overset{\to}{s} + \iint\limits_{S_2} \overset{\to}{a}\,d\overset{\to}{s} = \iint\limits_{S_1} \rot \: \overset{\to}{b}\,d\overset{\to}{s} + \iint\limits_{S_2} \rot \: \overset{\to}{b}\,d\overset{\to}{s} \myeq{\textit{Стокс}} \iint\limits_{\gamma_1} \overset{\to}{b}\,d\overset{\to}{r} + \iint\limits_{\gamma_2} \overset{\to}{b}\,d\overset{\to}{r} = \iint\limits_{\gamma_1} \overset{\to}{b}\,d\overset{\to}{r} - \iint\limits_{\gamma_1} \overset{\to}{b}\,d\overset{\to}{r} = 0$
\end{proof}

\begin{thm}
Если непрерывно-дифференцируемое векторное поле $\overset{\to}{a}(M), \; M \in G$ соленоидально в области $G \subset \bbR^3$, то 

\begin{multline}
\forall M \in G \; \exists \delta > 0, \exists \textit{непр-дифф. вект. поле } \overset{\to}{b}(P), \; P \in O_\delta(M): \\ \quad \forall P \in O_\delta(M): \; \overset{\to}{a}(P) = \rot \: \overset{\to}{b}(P) 
\end{multline}
\end{thm}

\begin{thm}[Гельмгольц] 
Любое непрерывно-дифференцируемое векторное поле $\overset{\to}{a}(M), \; M \in G$, $G$ -- область в $\bbR^3$, является суммой двух непрерывно-дифференцируемых векторных полей -- потенциального и соленоидального:

$$
\overset{\to}{a}(M) = \grad \: u(M) + \overset{\to}{c}(M), \quad \forall M \in G, \quad \Div \: \overset{\to}{c}(M) = 0
$$
\end{thm}