\chapter{Линейные обыкновенные дифференциальные уравнения с переменными коэффициентами. Фундаментальная система решений. Определитель Вронского. Формула Лиувилля-Остроградского.}

\section{Линейные обыкновенные дифференциальные уравнения с переменными коэффициентами}

\subsection{Общие свойства}

Линейным дифференциальным уравнением порядка $n$ называется уравненеие 

\begin{equation} \label{ch28eq1}
y^{(n)} + a_1(x)y^{(n - 1)} + \ldots + a_n(x)y = f(x),
\end{equation}

где $x \in [\alpha, \beta], \: a_j(x), \: j = \overline{1,n},$ -- заданные непрерывные функции на $[\alpha, \beta]$, называемые коэффициентами уравнения $(\ref{ch28eq1})$, и $f(x)$ --заданная непрерывная на $[\alpha, \beta]$ функция, называемая правой частью уравнения $(\ref{ch28eq1})$. При $f(x) \equiv 0$ на $[\alpha, \beta]$, уравнение -- однородное, в противном случае -- неоднородное. $a_j(x), \: j = \overline{1,n}, \; f(x)$ могут быть комплексными.

Функция $y = \phi(x)$ называется решением уравения $(\ref{ch28eq1})$ на $[\alpha, \beta]$, если $\phi(x) \; n$ раз непрерывно дифференцируема на $[\alpha, \beta]$ и обращает $(\ref{ch28eq1})$ в тождество на всем $[\alpha, \beta]$.

\begin{lemm}
Если $f(x) = f_1(x) + f_2(x)$ и $y_i(x)$ -- решение уравнения $(\ref{ch28eq1})$ при $f(x) \equiv f_i(x)$ на $[\alpha, \beta], \: i = 1,2$, то функция $y(x) = y_1(x) + y_2(x)$ является решением уравнеия $(\ref{ch28eq1})$.
\end{lemm}  

\begin{cons}
Если $y_1(x), y_2(x)$ -- решения линейного однородного уравнения и $c_1, c_2$ -- произвольные числ, то линейная комбинация $y = c_1y_1(x) + c_2y_2(x)$ также является решением линейного однородного уравнения.
\end{cons}

Решение уравнения $(\ref{ch28eq1})$ всегда можно свести к решению линейной системы дифференциальных уравнений порядка $n$ следующего вида:

\begin{equation} \label{ch28eq2}
y'(x) = A(x)y(x) + f(x),
\end{equation}
где

\begin{multline}
y(x) = \begin{pmatrix}
y_1(x) \\
\vdots \\
y_n(x)
\end{pmatrix},
A(x) = \begin{pmatrix}
0 & 0 & 0 & \ldots & 0 \\
0 & 0 & 1 & \ldots & 0 \\
\hdotsfor{5} \\
0 & 0 & 0 & \ldots & 1 \\
-a_n(x) & -a_{n - 1}(x) & -a_{n - 2}(x) & \ldots & -a_1(x)
\end{pmatrix}, \\
f(x) = \begin{pmatrix}
0 \\
\vdots \\ 
0 \\
f(x)
\end{pmatrix}.
\end{multline}

\begin{lemm} \label{ch28lemm2}
Уравнение $(\ref{ch28eq1})$ эквивалентно системе $(\ref{ch28eq2})$.
\end{lemm}

\begin{proof}
Пусть $y = \phi(x)$ -- решение $(\ref{ch28eq1})$. Положим $y_1(x) = \phi(x), \: y_2(x) = \phi'(x), \ldots, y_n(x) = \phi^{(n - 1)}(x)$. Тогда вектор-функция с компонентами $\phi(x), \phi'(x), \ldots, \phi^{(n - 1)}(x)$ удовлетворяет системе $(\ref{ch28eq2})$. Наоборот, если вектор-функция с компонентами $\phi(x), \phi'(x), \ldots, \phi^{(n - 1)}(x)$ -- решение системы $(\ref{ch28eq1})$, то, исключив из $(\ref{ch28eq2})$ переменные $y_2, \ldots, y_n$, получаем, что $y_1 = \phi(x)$ -- решение уравнения $(\ref{ch28eq1})$.
\end{proof}

Лемма $\ref{ch28lemm2}$ позволяет  перенести все результаты для линейных систем на случай уравнения $(\ref{ch28eq1})$.

Рассмотрим для уравнения $(\ref{ch28eq1})$ начальные условия

\begin{equation} \label{ch28eq3}
y(x_0) = y_1^{(0)}, \; y'(x_0) = y_2^{(0)}, \; \ldots, \; y^{(n - 1)}(x_0) = y_n^{(0)},
\end{equation}

где $x_0 \in [\alpha, \beta]$ и $y_1^{(0)}, \ldots, y_n^{(0)}$ -- заданные числа.

\begin{thm}
Пусть все функции $a_j(x), \: j = \overline{1,n}$ и $f(x)$ -- непрерывны на $[\alpha, \beta]$ и пусть $x_0 \in [\alpha, \beta]$. Тогда при произвольных начальных значениях $y_1^{(0)}, \ldots, y_n^{(0)}$ решение задачи Коши $(\ref{ch28eq1}), \: (\ref{ch28eq3})$ существует и единственно на всем $[\alpha, \beta]$.
\end{thm}

\begin{proof}
Сделав замену 
$$
y_1(x) = y(x), \; y_2(x) = y'(x), \; \ldots, \; y_n(x) = y^{(n - 1)}(x),
$$

сведем уравнение $(\ref{ch28eq1})$ к системе $(\ref{ch28eq2})$. При этом начальные условия примут вид

\begin{equation} \label{ch28eq4}
y(x_0) = y^{(0)},
\end{equation}
где $y^{(0)}$ -- вектор с компонентами $y_1^{(0)}, \ldots, y_n^{(0)}$. В силу леммы $\ref{ch28lemm2}$ задача Коши $(\ref{ch28eq1}), \: (\ref{ch28eq3})$ эквивалентна задаче Коши $(\ref{ch28eq2}), \: (\ref{ch28eq4})$. В силу условий теоремы $A(x)$ и $f(x)$ -- непрерывны на $[\alpha, \beta]$. Следовательно, для задачи Коши $(\ref{ch28eq2}), \: (\ref{ch28eq4})$ выполнены все условия теоремы о существовании и единственности задачи Коши для линейной системы уравнений. Значит, и решение задачи Коши $(\ref{ch28eq1}), \: (\ref{ch28eq3})$ существует и единственно на $[\alpha, \beta]$.
\end{proof}
\section{Фундаментальная система решений}
\section{Определитель Вронского}
\section{Формула Лиувилля-Остроградского}