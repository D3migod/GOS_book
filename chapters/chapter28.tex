\chapter{Линейные обыкновенные дифференциальные уравнения с переменными коэффициентами. Фундаментальная система решений. Формула Лиувилля-Остроградского. Определитель Вронского.}

\section{Линейные обыкновенные дифференциальные уравнения с переменными коэффициентами}

Линейным дифференциальным уравнением порядка $n$ называется уравненеие 

\begin{equation} \label{ch28.1eq1}
y^{(n)} + a_1(x)y^{(n - 1)} + \ldots + a_n(x)y = f(x),
\end{equation}

где $x \in [\alpha, \beta], \: a_j(x), \: j = \overline{1,n},$ -- заданные непрерывные функции на $[\alpha, \beta]$, называемые коэффициентами уравнения $(\ref{ch28.1eq1})$, и $f(x)$ --заданная непрерывная на $[\alpha, \beta]$ функция, называемая правой частью уравнения $(\ref{ch28.1eq1})$. При $f(x) \equiv 0$ на $[\alpha, \beta]$, уравнение -- однородное, в противном случае -- неоднородное. $a_j(x), \: j = \overline{1,n}, \; f(x)$ могут быть комплексными.

Функция $y = \phi(x)$ называется решением уравения $(\ref{ch28.1eq1})$ на $[\alpha, \beta]$, если $\phi(x) \; n$ раз непрерывно дифференцируема на $[\alpha, \beta]$ и обращает $(\ref{ch28.1eq1})$ в тождество на всем $[\alpha, \beta]$.

\begin{lemm}
Если $f(x) = f_1(x) + f_2(x)$ и $y_i(x)$ -- решение уравнения $(\ref{ch28.1eq1})$ при $f(x) \equiv f_i(x)$ на $[\alpha, \beta], \: i = 1,2$, то функция $y(x) = y_1(x) + y_2(x)$ является решением уравнеия $(\ref{ch28.1eq1})$.
\end{lemm}  

\begin{cons}
Если $y_1(x), y_2(x)$ -- решения линейного однородного уравнения и $c_1, c_2$ -- произвольные числ, то линейная комбинация $y = c_1y_1(x) + c_2y_2(x)$ также является решением линейного однородного уравнения.
\end{cons}

Решение уравнения $(\ref{ch28.1eq1})$ всегда можно свести к решению линейной системы дифференциальных уравнений порядка $n$ следующего вида:

\begin{equation} \label{ch28.1eq2}
y'(x) = A(x)y(x) + f(x),
\end{equation}
где

\begin{multline}
y(x) = \begin{pmatrix}
y_1(x) \\
\vdots \\
y_n(x)
\end{pmatrix},
A(x) = \begin{pmatrix}
0 & 1 & 0 & \ldots & 0 \\
0 & 0 & 1 & \ldots & 0 \\
\hdotsfor{5} \\
0 & 0 & 0 & \ldots & 1 \\
-a_n(x) & -a_{n - 1}(x) & -a_{n - 2}(x) & \ldots & -a_1(x)
\end{pmatrix}, \\
f(x) = \begin{pmatrix}
0 \\
\vdots \\ 
0 \\
f(x)
\end{pmatrix}.
\end{multline}

\begin{lemm} \label{ch28.1lemm2}
Уравнение $(\ref{ch28.1eq1})$ эквивалентно системе $(\ref{ch28.1eq2})$.
\end{lemm}

\begin{proof}
Пусть $y = \phi(x)$ -- решение $(\ref{ch28.1eq1})$. Положим $y_1(x) = \phi(x), \: y_2(x) = \phi'(x), \ldots, y_n(x) = \phi^{(n - 1)}(x)$. Тогда вектор-функция с компонентами $\phi(x), \phi'(x), \ldots, \phi^{(n - 1)}(x)$ удовлетворяет системе $(\ref{ch28.1eq2})$. Наоборот, если вектор-функция с компонентами $\phi(x), \phi'(x), \ldots, \phi^{(n - 1)}(x)$ -- решение системы $(\ref{ch28.1eq1})$, то, исключив из $(\ref{ch28.1eq2})$ переменные $y_2, \ldots, y_n$, получаем, что $y_1 = \phi(x)$ -- решение уравнения $(\ref{ch28.1eq1})$.
\end{proof}

Лемма $\ref{ch28.1lemm2}$ позволяет  перенести все результаты для линейных систем на случай уравнения $(\ref{ch28.1eq1})$.

Рассмотрим для уравнения $(\ref{ch28.1eq1})$ начальные условия

\begin{equation} \label{ch28.1eq3}
y(x_0) = y_1^{(0)}, \; y'(x_0) = y_2^{(0)}, \; \ldots, \; y^{(n - 1)}(x_0) = y_n^{(0)},
\end{equation}

где $x_0 \in [\alpha, \beta]$ и $y_1^{(0)}, \ldots, y_n^{(0)}$ -- заданные числа.

\begin{thm}
Пусть все функции $a_j(x), \: j = \overline{1,n}$ и $f(x)$ -- непрерывны на $[\alpha, \beta]$ и пусть $x_0 \in [\alpha, \beta]$. Тогда при произвольных начальных значениях $y_1^{(0)}, \ldots, y_n^{(0)}$ решение задачи Коши $(\ref{ch28.1eq1}), \: (\ref{ch28.1eq3})$ существует и единственно на всем $[\alpha, \beta]$.
\end{thm}

\begin{proof}
Сделав замену 
$$
y_1(x) = y(x), \; y_2(x) = y'(x), \; \ldots, \; y_n(x) = y^{(n - 1)}(x),
$$

сведем уравнение $(\ref{ch28.1eq1})$ к системе $(\ref{ch28.1eq2})$. При этом начальные условия примут вид

\begin{equation} \label{ch28.1eq4}
y(x_0) = y^{(0)},
\end{equation}
где $y^{(0)}$ -- вектор с компонентами $y_1^{(0)}, \ldots, y_n^{(0)}$. В силу леммы $\ref{ch28.1lemm2}$ задача Коши $(\ref{ch28.1eq1}), \: (\ref{ch28.1eq3})$ эквивалентна задаче Коши $(\ref{ch28.1eq2}), \: (\ref{ch28.1eq4})$. В силу условий теоремы $A(x)$ и $f(x)$ -- непрерывны на $[\alpha, \beta]$. Следовательно, для задачи Коши $(\ref{ch28.1eq2}), \: (\ref{ch28.1eq4})$ выполнены все условия теоремы о существовании и единственности задачи Коши для линейной системы уравнений. Значит, и решение задачи Коши $(\ref{ch28.1eq1}), \: (\ref{ch28.1eq3})$ существует и единственно на $[\alpha, \beta]$.
\end{proof}
\section{Фундаментальная система решений}

Рассмотрим линейное однородное уравнение порядка $n$

\begin{equation} \label{ch28.2eq1}
y^{(n)} + a_1(x)y^{(n - 1)} + \ldots + a_n(x)y = 0,
\end{equation} 

где $a_j(x), \: j = \overline{1,n}$, заданные непрерывные функции на $[\alpha, \beta]$.

\begin{defn}
Решения $y_1(x), \ldots, y_k(x)$ уравнения $(\ref{ch28.2eq1})$ называется линейно зависимым на $[\alpha, \beta]$, если $\exists$ числа $c_1, \ldots, c_k$, одновременно не равные нулю и такие, что 

$$
c_1y_1(x) + \ldots + c_ky_k(x) \equiv 0, \quad \forall x \in [\alpha, \beta].
$$

В противном случае решения $y_1(x), \ldots, y_k(x)$ называаются линейно независимыми на $[\alpha, \beta]$.
\end{defn}

Рассмотрим линейную однородную систему, которая эквивалентна уравнению $(\ref{ch28.1eq1})$:

\begin{multline} \label{ch28.2eq2}
y'(x) = A(x)y(x), \\
y(x) = \begin{pmatrix}
y_1(x) \\
\vdots \\
y_n(x)
\end{pmatrix},
A(x) = \begin{pmatrix}
0 & 1 & 0 & \ldots & 0 \\
0 & 0 & 1 & \ldots & 0 \\
\hdotsfor{5} \\
0 & 0 & 0 & \ldots & 1 \\
-a_n & -a_{n - 1} & -a_{n - 2} & \ldots & -a_1
\end{pmatrix}.
\end{multline}

\begin{lemm} \label{ch28.2lemm1}
Решения $y_1(x), \ldots, y_k(x)$ уравнения $\ref{ch28.2eq1}$ линейно зависимы на $[\alpha, \beta]$ тогда и только тогда, когда соответствующие им решения $y_1(x), \ldots, y_k(x)$ системы $\ref{ch28.2eq2}$ линейно зависимы на $[\alpha,\beta]$ (здесь $y_j(x)$ -- вектор-функция с компонентами $y_j(x), y'_j(x), \ldots, y^{(n - 1)}_j(x), \: j = \overline{1,k})$.
\end{lemm}

\begin{proof}
Пусть решения $y_1(x), \ldots, y_k(x)$ уравнения $(\ref{ch28.2eq1})$ линейно зависимы на $[\alpha, \beta]$. Тогда найдутся такие числа $c_1, \ldots, c_k, \: |c_1| + \ldots + |c_k| > 0$, что 

$$
c_1y_1(x) + \ldots + c_ky_k(x) \equiv 0, \quad \forall x \in [\alpha, \beta].
$$

Дифференцируя последовательно это тождество $(n - 1)$ раз, получаем тождество на $[\alpha, \beta]$ для решений системы $(\ref{ch28.2eq2})$:

$$
c_1y_1(x) + \ldots + c_ky_k(x) \equiv 0,
$$
т.е. решения $y_1(x), \ldots, y_k(x)$ системы $(\ref{ch28.2eq2})$ линейно зависимы на $[\alpha, \beta]$.

Обратно, если выполнено последнее тождество на $[\alpha, \beta]$ с некоторыми, одновременно не равными нулю, числами $c_1, \ldots, c_k$, то первая компонента этого векторного тождества означает линейную зависимость решений $y_1(x), \ldots, y_k(x)$ уравнения $(\ref{ch28.2eq1}).$
\end{proof}

\begin{cons}
Решения $y_1(x), \ldots, y_k(x)$ уравнения $(\ref{ch28.2eq1})$ линейно независимы на $[\alpha, \beta]$ тогда и только тогда, когда решения $y_1(x), \ldots, y_k(x)$ системы $(\ref{ch28.2eq2})$ линейно независимы на $[\alpha, \beta]$.
\end{cons}

\begin{defn}
Совокупность произвольных $n$ независимых решений $\phi_1(x), \ldots, \phi_n(x)$ уравнения $(\ref{ch28.2eq1})$ называется фундаментальной системой решений уравнения $(\ref{ch28.2eq1})$.
\end{defn}

Из леммы $\ref{ch28.2lemm1}$ в качестве следствия получаем следующее утверждение.

\begin{lemm} \label{ch28.2lemm2}
Решения $\phi_1(x), \ldots, \phi_n(x)$ уравнения $(\ref{ch28.2eq1})$ образуют функдаментальную систему решений уравнения $(\ref{ch28.2eq1})$ в том и только в том случае, когда вектор-функция $\Phi_j(x)$ с компонентами $\phi_j(x), \ldots, \phi^{(n - 1)_j(x)}, \:  = \overline{1,n}$, образуют фундаментальную систему решений линейной однородной системы $\ref{ch28.2eq2}$.
\end{lemm}

С помощью леммы $\ref{ch28.2lemm2}$ все утверждения о фундаментальных системах решений линейной однородной системы переносятся на фундаментальные системы решений линейных однородных уравнений порядка $n$.

\begin{thm} \label{ch28.2thm1}
Для уравнения $(\ref{ch28.2eq1})$ существует бесконечное множество фундаментальных систем решений.
\end{thm}

\begin{proof}
Уравнение $(\ref{ch28.2eq1})$ эквиваленто системе $(\ref{ch28.2eq2})$, для которой справедлив аналог теоремы $\ref{ch28.2thm1}$ для линейной системы с переменными коэффициентами. В силу леммы $\ref{ch28.2lemm2}$ тогда справедлива и теорема $\ref{ch28.2thm1}$.
\end{proof}

\begin{thm} \label{ch28.2thm2}
Если $\phi_1(x), \ldots, \phi_n(x)$ -- фундаментальная система решений уравнения $(\ref{ch28.2eq1})$, то каждое решение $y(x)$ уравнения $(\ref{ch28.2eq1})$ представимо единственным образом в виде

$$
y(x) = c_1\phi_1(x) + \cdots + c_n\phi_n(x),
$$
где $c_1, \ldots, c_n$ -- постоянные.
\end{thm}

\begin{proof}
По лемме $\ref{ch28.2lemm2}$ вектор-функции $\Phi_j(x)$ с компонентами $\phi_j(x), \phi'_j(x), \ldots, \phi^{(n - 1)}_j(x), \: j = \overline{1,n}$, образуют фундаментальную систему решений системы $(\ref{ch28.2eq2})$, эквивалетной уравнению $(\ref{ch28.2eq1})$. По теореме, аналогичной теореме $\ref{ch28.2thm2}$, для линейной системы с переменными коэффициентами любое решение $y(x)$ системы $(\ref{ch28.2eq2})$ единственным образом представимо в виде

$$
y(x) = c_1 \Phi_1(x) + \cdots + c_n \Phi_n(x).
$$
Первая строка этого векторного равенства и дает утверждение теоремы $\ref{ch28.2thm2}$.
\end{proof}
\section{Определитель Вронского}
\section{Формула Лиувилля-Остроградского}