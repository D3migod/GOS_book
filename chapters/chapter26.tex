\chapter{Линейные обыкновенные дифференциальные уравнения с постоянными коэффициентами и правой частью — квазимногочленом.}

\section{Линейные обыкновенные дифференциальные уравнения с постоянными коэффициентами}

Дифференциальное уравнение вида
\begin{equation}\label{diff.1}
y^{(n)}(x)+a_1y^{(n-1)}(x)+\dots+a_{n-1}y'(x)+a_ny(x)=0,
\end{equation}
где $x\in \bbR$, и $a_1,\dots,a_n$ -- заданные действительные или комплексные числа, называют линейным однородным дифференциальным уравнением порядка $n$ с постоянными коэффициентами. Числа $a_1,\dots,a_n$ называют коэффициентами уравнения \eqref{diff.1}.

С помощью дифференциального многочлена 
$$
L(D)= D^n+a_1D^{n-1}+\dots+a_{n-1}D+a_n
$$
уравнение eqref(diff.1} коротко записывается в виде 
\begin{equation}
L(D)y(x)=0.
\end{equation}
\subsection{Общее решение}
