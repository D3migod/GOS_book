\chapter{Линейные обыкновенные дифференциальные уравнения с постоянными коэффициентами и правой частью — квазимногочленом.}
\section{Дифференциальные многочлены и общий метод решения линейных уравнений с постоянными коэффициентами}
Обозначим через $C^1(\bbR)$ множество всех комплекснозначных функций, заданных и непрерывных на всех числовой оси $\bbR$, а через $C^k(\bbR)$, $k\in \bbN$, --- множество всех функций, $k$ раз непрерывно дифференцируемых на всей оси $\bbR$.
\begin{defn} \label{ch26.1defn1}
Говорят, что задан \textit{оператор дифференцирования $D$}, действующий из $C^1(\bbR)$ в $C(\bbR)$, если каждой функции $y(x)\in C^1(\bbR)$ оператор $D$ ставит в соответствие функцию $y'(x) \in C(\bbR)$ по формуле $Dy(x)=y'(x)$.
\end{defn}

\begin{defn} \label{ch26.1defn2}
\textit{$k$-я степень оператора дифференцирования $D^k$}, $k\in \bbN$, является оператором, действующим из множества $C^k(\bbR)$ во множество $C(\bbR)$ по формуле $D^ky(x)=y^{(k)}(x)$.
\end{defn}

\begin{defn} \label{ch26.1defn3}
\textit{Дифференциальным многочленом степени} $n \in \bbN$ (или многочленом степени $n$ от оператора дифференцирования $D$) $L(D) = D^n + a_1D^{n-1}+\cdots+a_{n-1}D + a_n$, где $a_1, \dots , a_n$ --- заданные числа (действительные или комплексные), называют оператор, действующий из множества $C^n(\bbR)$ во множество $C(\bbR)$ по формуле $L(D)y(x)=y^{(n)}+a_1y^{(n-1)}+\cdots+a_ny.$
\end{defn}

\begin{lemm} \label{ch26lemm1.1}
Дифференциальный многочлен степени $n$ является линейным оператором, т.е. для любых функций $y_1(x),y_2(x)\in C^n(\bbR)$ и любых чисел $c_1,c_2$ выполнено равенство
$$
L(D)[c_1y_1(x)+c_2y_2(x)]=c_1L(D)y_1(x)+c_2L(D)y_2(x).
$$ 
\end{lemm}

\begin{proof}
Требуемое утверждение получается из определения $\ref{ch26.1defn3}$ дифференциального многочлена и свойства линейности для производных. Действительно,
\begin{multline*}
L(D)(c_1y_1+c_2y_2)=\\=(c_1y_1+c_2y_2)^{(n)}+a_1(c_1y_1+c_2y_2)^{(n-1)}+...+a_n(c_1y_1+c_2y_2)=\\=c_1(y_1^{(n)}+a_1y_1^{(n-1)}+...+a_ny_1)+c_2(y_2^{(n)}+a_1y_2^{(n-1)}+...+a_ny_2)=\\=c_1L(D)y_1+c_2L(D)y_2.
\end{multline*}
\end{proof}

\begin{lemm} \label{ch26lemm1.2}
Если $\lambda$ -- комплексное число, то для любой $y(x)\in C^n(\bbR)$ справедлива так называемая формула сдвига
$$
L(D)\left[e^{\lambda x}\cdot y(x)\right]=e^{\lambda x}\cdot L(D+\lambda)y(x).
$$
\end{lemm}
\begin{proof}
При любом $k\in \bbN$ по формуле Лейбница получаем
\begin{multline*}
D^k[e^{\lambda x}\cdot y]=(e^{\lambda x}\cdot y)^{(k)}=\sum_{j=1}^{k}\limits C_k^j(e^{\lambda x})^{(j)}\cdot y^{(k-j)}=\\=e^{\lambda x}\cdot \sum_{j=0}^{k}\limits C_k^j\lambda^jD^{k-j}y=e^{\lambda x}(D+\lambda)^ky.
\end{multline*}
В силу этого,
\begin{multline*}
L(D)[e^{\lambda x}y]=e^{\lambda x}(D+\lambda)^ny+a_1e^{\lambda x}(D+\lambda)^{n-1}y+...+a_ne^{\lambda x}y=\\=e^{\lambda x}L(D+\lambda)y.
\end{multline*}
\end{proof}

\begin{lemm} \label{ch26lemm1.3}
Все решения уравнения $z'-\lambda z = f(x)$, где $\lambda$ -- комплексное число и $f(x)$ -- заданная комплекснозначная функция из $C(\bbR)$, задаются формулой
	
$$
z(x)=e^{\lambda x}\left(C+ \int_{x_0}^{x}\limits e^{-\lambda\zeta}\cdot f(\zeta)d\zeta\right),
$$
где $C$ -- произвольная комплексная постоянная. 
\end{lemm}

\begin{proof}
Ищем решение уравнения в виде $z(x)=c(x)e^{\lambda x}$. После подстановки $z(x)$ в уравнение и упрощений получаем 
$$
c'(x)=e^{-\lambda x}\cdot f(x).
$$
Отсюда
$$
c(x)=C+\int_{x_0}^{x}\limits e^{-\lambda\zeta}\cdot f(\zeta)d\zeta,
$$
что и доказыает лемму.
\end{proof}

\section{Линейные однородные дифференциальные уравнения порядка $n$ с постоянными коэффициентами}

Дифференциальное уравнение вида
\begin{equation}\label{ch26diff.1}
y^{(n)}(x)+a_1y^{(n-1)}(x)+\dots+a_{n-1}y'(x)+a_ny(x)=0,
\end{equation}
где $x\in \bbR$, и $a_1,\dots,a_n$ -- заданные действительные или комплексные числа, называют линейным однородным дифференциальным уравнением порядка $n$ с постоянными коэффициентами. Числа $a_1,\dots,a_n$ называют коэффициентами уравнения \eqref{ch26diff.1}.
С помощью дифференциального многочлена 
$$
L(D)= D^n+a_1D^{n-1}+\dots+a_{n-1}D+a_n
$$
уравнение \eqref{ch26diff.1} коротко записывается в виде
\begin{equation} \label{ch26diff.2}
L(D)y(x)=0.
\end{equation} 

\begin{lemm} \label{ch26lemm2.1}
Если $y_1(x)$, $y_2(x)$ -- какие-либо решения уравнения \eqref{ch26diff.1}  и $C_1$, $C_2$ -- произвольные комплексные числа, то функция $y=C_1y_1(x)+C_2y_2(x)$ также является решением уравнения \eqref{ch26diff.1}.
\end{lemm}

\begin{proof}
Воспользуемся формой \eqref{ch26diff.2} записи уравнения \eqref{ch26diff.1}. В силу линейности многочлена $L(D)$ (см. лемму \ref{ch26lemm1.1}), имеем
$$
L(D)y=L(D)(C_1y_1+C_2y_2)=C_1L(D)y_1+C_2L(D)y_2=0,
$$
так как $L(D)y_1=L(D)y_2=0$ по условию леммы
\end{proof}

Рассмотрим функции вида
\begin{equation} \label{ch26diff.3}
\phi(x)=P_1(x)e^{\lambda_1x}+...+P_m(x)e^{\lambda_mx},
\end{equation}
где $\lambda_1,...,\lambda_m$ -- попарно различные комплексные числа, а $P_1(x),...,P_m(x)$ -- многочлены с комплексными коэффициентами.


\begin{lemm}\label{ch26lemm2.2}
Если в \eqref{ch26diff.3} $\phi(x)=0$ для всех $x \in \bbR$, то коэффициенты во всех многочленах $P_1(x),...,P_m(x)$ нулевые.
\end{lemm}


\begin{proof}
Применим индукцию по $m$. При $m=1$ утверждение леммы очевидно. Пусть утверждение леммы справледливо, если в формуле \eqref{ch26diff.3} заменить $m$ на $(m-1)$. При $m>1$ рассмотрим функцию

$$
\psi(x)=e^{-\lambda_1x}\cdot\phi(x)=P_1(x)+ \sum_{k=2}^{m}\limits P_k(x)e^{(\lambda_k-\lambda_1)x}=0.
$$
Продифференцируем $\psi(x)$ $(N+1)$ раз, где $N$ -- степень многочлена $P_1(x)$. В силу того, что $P_1^{(N+1)}(x)=0$, получим 
$$
\sum_{k=2}^{m}\limits\left[P_k(x)e^{(\lambda_k-\lambda_1)x}\right]^{(N+1)}=0
$$
или
$$
\sum_{k=2}^{m}\limits Q_k(x)e^{(\lambda_k-\lambda_1)x}=0,
$$
где $Q_k$ -- многочлены той же степени, что и $P_k(x)$, так как $\lambda_k-\lambda_1 \neq 0$ при всех $k=\overline{2,m}$. Из предположения индукции $Q_k(x)\equiv 0,\,\fa k=\overline{2,m}$. Следовательно, $P_k(x)\equiv 0,\,$ $k=\overline{2,m}$. Тогда и $P_1(x)\equiv 0$. Это значит, что все коэффициенты многочленов $P_1(x),...,P_m(x)$ в \eqref{ch26diff.3} нулевые.
\end{proof}

Рассмотрим характеристический многочлен
$$
L(\lambda)=\lambda^n+a_1\lambda^{n-1}+...+a_n.
$$
Уравнение $L(\lambda)=0$ называется \textit{характеристическим уравнением} для \eqref{ch26diff.1}.

Напомним, что число $\lambda_0$ называется \textit{корнем кратности} $k\,(k\in\bbN, 1\leq k \leq n$) уравнения $L(\lambda)=0$, если
$$
L(\lambda)=(\lambda-\lambda_0)^k\cdot\L_1(\lambda),
$$
где $L_1(\lambda)$ -- многочлен степени $(n-k)$ и $L_1(\lambda_0)\neq 0$. Из формулы Тейлора для $L(\lambda)$ при $\lambda=\lambda_0$ сразу следует, что  $\lambda_0$ -- корень крастности $k$ для $L(\lambda)=0$ тогда и только тогда, когда
$$
L(\lambda_0)=L'(\lambda_0)=...=L^{(k-1)}(\lambda_0)=0, \qquad L^{(k)}(\lambda_0)\neq 0
$$




\begin{lemm} \label{ch26lemm2.3}
Если $\lambda_0$ -- корень кратности $k$ характеристического уравнения $L(\lambda)=0$, то каждая из функций 
$$
e^{\lambda_0 x},xe^{\lambda_0 x},...,x^{k-1}e^{\lambda_0 x}
$$
является решением уравнения \eqref{ch26diff.1}
\end{lemm}

\begin{proof}

a) $\lambda_0=0$. Тогда $L(\lambda)=\lambda^k(\lambda^{n-k}+a_1\lambda^{n-k-1}+\cdots+a_{n-k})$, где  $a_{n-k}\neq 0$, и, следовательно,
$$
L(D)=D^n+a_1D^{n-1}+...+a_{n-k}D^k
$$
Нетрудно проверить, что функциии $1,x,...,x^{k-1}$ являются решениями $L(D)y=0$.

б) $\lambda_0\neq 0$. Сделаем замену $y=e^{\lambda_0x}z$. По формуле сдвига (см. лемму \ref{ch26lemm1.2})
$$
L(D)y=e^{\lambda_0x}L(D+\lambda_0)z=0.
$$
Характеристический многочлен $L(\lambda+\lambda_0)$ имеет корень $\lambda=0$ кратности $k$. В силу п. a) уравнение $L(D+\lambda_0)z=0$ имеет решения $1,x,...,x^{k-1}$. Из замены получаем утверждение леммы.
\end{proof}


\begin{thm} \label{ch26thm1}
Пусть характеристическое уравнение $L(\lambda)=0$ имеет корни $\lambda_1,...,\lambda_m\;(m\in \bbN, 1\leq m\leq n)$ соответственно кратности $k_1,...,k_m\;(k_1+...+k_m=n).$ Тогда:

а) любая функция вида
\begin{equation} \label{ch26diff.4}
y(x)=P_1(x)e^{\lambda_1x}+...+P_m(x)e^{{\lambda_m}x},
\end{equation}
где $P_j(x)=C_0^j+C_1^jx+...+C_{k_j-1}^jx^{k_j-1}$ -- многочлен степени $(k_j-1)$, коэффициентами которого служат произвольные комплексные постоянные  $C_0^j,...,C_{k_j-1}^j$, является решением уравнения \eqref{ch26diff.1};

б) если $y(x)$ -- какое-либо решение уравнения \eqref{ch26diff.1}, то найдется единственный набор коэффициентов многочленов $P_1(x),...,P_m(x)$, при котором это решение $y(x)$ задается формулой \eqref{ch26diff.4}.
\end{thm}

\begin{proof}

п. a) теоремы немедленно следует из леммы \ref{ch26lemm2.3} и принципа суперпозиции для уравнения \eqref{ch26diff.1} (см. лемму \ref{ch26lemm2.1}).

п. б) докажем методом математической индукции по $n$. Пусть $y(x)$ -- какое-либо решение \eqref{ch26diff.1}. При $n=1$ уравнение \eqref{ch26diff.1} имеет вид $y'+a_1y=0$  и по лемме \ref{ch26lemm1.3} все его решения имеют вид $y=Ce^{-a_1x}$. Ясно, что при некотором единственном значении $C$ эта формула содержит и наше решение. Пусть теперь $n>1$  и пусть всякое решение $y(x)$ линейного однородного уравнения порядка $(n-1)$ с постоянными коэффициентами единственным образом записывается в форме \eqref{ch26diff.4} с заменой $n$ на $(n-1)$.
В силу условий теоремы 
$$
L(\lambda)=(\lambda-\lambda_1)^{k_1}(\lambda-\lambda_2)^{k_2}...(\lambda-\lambda_m)^{k_m}.
$$
Значит,
$$
L(D)=(D-\lambda_1)^{k_1}(D-\lambda_2)^{k_2}...(D-\lambda_m)^{k_m}.
$$
Введем дифференциальный многочлен степени $(n-1)$
$$
M(D)=(D-\lambda_1)^{k_1-1}(D-\lambda_2)^{k_2}...(D-\lambda_m)^{k_m},
$$
где при $k_1=1$ первый сомножитель отсутсвует. Тогда $L(D)=M(D)(D-\lambda_1)$. Положим $(D-\lambda_1)y=z$. В таком случае уравнение \eqref{ch26diff.2} эквивалентно системе
\begin{equation}\label{ch26diff.5}
\begin{cases}
(D-\lambda_1)y=z,\\M(D)z=0.
\end{cases}
\end{equation}
Каждое решение второго уравнения системы \eqref{ch26diff.5} в силу предположения индукции имеет вид
$$
z(x)=Q_1(x)e^{\lambda_1x}+Q_2(x)e^{\lambda_2x}+...+Q_m(x)e^{\lambda_mx},\quad 1\leq m\leq n,
$$
где $Q_1(x)$ -- многочлен степени $(k_1-2)$ в случае $k_1>1$ и $Q_1(x)\equiv 0$ в случае $k_1=1$, а $Q_j(x)$ -- многочлены степени $(k_j-1)$ при всех $j=\overline{2,m}$. По лемме \ref{ch26lemm1.3} решение первого уравнения системы \eqref{ch26diff.5} имеет вид
\begin{equation}\label{ch26diff.6}
y=e^{\lambda_1x}\left(C+\int e^{-\lambda_1x} z(x) dx\right),
\end{equation}
где $C$ -- комплексная постоянная.

Учитывая, что при целом $l\geq 0$ первообразная
$$
\int x^le^{\lambda x}dx = \begin{cases}
(b_0x^l+...+b_l)e^{\lambda x},\quad &\lambda \neq 0,\quad  b_0\neq 0,\\
\frac{x^{l+1}}{l+1},\quad &\lambda=0.
\end{cases}
$$
и что $\lambda_j-\lambda_1\neq 0,\; \fa j=\overline{2,m}$, из вида $z(x)$ находим, что

$$
\int e^{-\lambda_1x}z(x)dx = \begin{cases}
P_1(x)+P_2(x)e^{(\lambda_2-\lambda_1)x}+...+P_m(x)e^{(\lambda_m-\lambda_1)x},\quad &k_1>1\\
P_2(x)e^{(\lambda_2-\lambda_1)x}+...+P_m(x)e^{(\lambda_m-\lambda_1)x},\quad &k_1=1
\end{cases}
$$
Подставляя это выражение в \eqref{ch26diff.6}, получаем, что рассматриваемое решение $y(x)$ уравнения \eqref{ch26diff.1} имеет вид \eqref{ch26diff.4}.

Рассуждением от противного установим единственность записи \eqref{ch26diff.4} для каждого уравнения \eqref{ch26diff.1}. Если существует решение $y(x)$ уравнения \eqref{ch26diff.1}, для которого
$$
y(x)=\sum_{k=1}^{m}\limits P_k(x)e^{\lambda_kx}=\sum_{k=1}^{m}\limits \tilde{P}_k(x)e^{\lambda_kx},
$$
то отсюда
$$
\sum_{k=1}^{m}\limits\left[P_k(x)-\tilde{P}_k(x)\right]e^{\lambda_kx}\equiv 0.
$$
Из леммы \ref{ch26lemm2.2} тогда получаем, что $P_k(x)\equiv\tilde{P}_k(x),\; \fa k= \overline{1,m}$
\end{proof}

\section{Линейные неоднородные уравнения c постоянными коэффициентами и правой частью -- квазимногочленом}
Эти уравнения имеют вид
\begin{equation} \label{ch26adiff1.1}
y^{(n)}(x)+a_1y^{(n-1)}(x)+...+a_ny(x)=f(x),
\end{equation}
где $a_1,...,a_n$ -- заданные комплексные или действительные числа, а правая часть $f(x)$ уравнения \eqref{ch26adiff1.1} -- заданная непрерывная функция на некотором промежутке $X$ оси $\bbR$.

Рассмотрим линейное однородное уравнение
\begin{equation}\label{ch26adiff1.2}
z^{(n)}(x)+a_1z^{(n-1)}(x)+...+a_nz(x)=0
\end{equation}
Прежде всего покажем, что если известно какое-либо решение $y_0(x)$ линейного неоднородного уравнения \eqref{ch26adiff1.1}, то замена $y(x)=z(x)+y_0(x)$ приводит уравнение \eqref{ch26adiff1.1} к линейному однородному уравнению \eqref{ch26adiff1.2}. Действительно, воспользовавшись представлением левой части \eqref{ch26adiff1.1} через дифференциальный многочлен
\begin{equation}\label{ch26adiff1.3}
L(D)=D^n+a_1D^{n-1}+...+a_n,
\end{equation}
получаем, что
$$
L(D)y=L(D)(z+y_0)=L(D)z+L(D)y_0=L(D)z+f(x)=f(x).
$$
Отсюда следует $L(D)z=0$, т.е. $z(x)$ -- решение \eqref{ch26adiff1.2}.

Это замечаение позволяет написать формулу всех решений линейного неоднородного уравнения \eqref{ch26adiff1.1}, если найти каким-то образом его решение $y_0(x)$, так как формула общего решения \eqref{ch26adiff1.2} была уже получена ранее. Именно, если $z_1(x),...,z_n(x)$ -- базис решений \eqref{ch26adiff1.2}, то формула $y=C_1z_1(x)+...+C_nz_n(x)+y_0(x)$ дает все решения \eqref{ch26adiff1.1}. Ее называют формулой общего решения \eqref{ch26adiff1.1}.

\begin{lemm}
Пусть $f(x)=f_1(x)+f_2(x)$ и пусть $y_1(x)$ -- какое-либо решение уравнения \eqref{ch26adiff1.1} при $f(x)\equiv f_1(x)$ и $y_2(x)$ -- какое-либо решение уравнения $\eqref{ch26adiff1.1}$ при $f(x) \equiv f_2(x)$.

Тогда $y(x)=y_1(x)+y_2(x)$ является решением уравнения \eqref{ch26adiff1.1}.
\end{lemm}

\begin{proof}
$$
L(D)y=L(D)(y_1+y_2)=L(D)y_1+L(D)y_2=f_1(x)+f_2(x)=f(x).
$$
\end{proof}

\begin{defn}
\textit{Квазимногочленом} называется функция $f(x)=e^{\mu x}P_m(x)$, где $\mu$ -- заданное комплексное число, $P_m(x)$ -- заданный многочлен степени $m$ с комплексными коэффициентами.
\end{defn}

Из теоремы \ref{ch26thm1} следует, что всякое решение линейного однородного уравнения с постоянными коэффициентами представляет собой конечную сумму квазимногочленов.

Покажем, в каком виде нужно искать частное решение линейного неоднородного уравнения с постоянными коэффициентами \eqref{ch26adiff1.1} с квазимногочленом в правой части. Найдя это частное решение и базис пространства решений \eqref{ch26adiff1.2}, немедленно получаем общее решение \eqref{ch26adiff1.1}.

Рассмотрим уравнение
\begin{equation}\label{ch26adiff1.4}
L(D)y(x)=e^{\mu x}\cdot P_m(x),
\end{equation}
где $\mu$ -- заданное комплексное число, а $P_m(x)$ -- заданный многочлен степени $m$.

\begin{defn}
Если число $\mu$ является корнем характеристического уравнения
$$
L(\lambda)\equiv \lambda^n+a_1\lambda^{n-1}+...+a_n=0,
$$
то говорят, что в уравнении \eqref{ch26adiff1.4} имеет место \textit{резонансный случай}. Если же $\mu$ не является корнем $L(\lambda)=0$, то говорят, что в уравнениии \eqref{ch26adiff1.4} имеет место \textit{нерезонансный случай}.
\end{defn}

\begin{thm} 
Для уравнения \eqref{ch26adiff1.4} существует и единственно решение вида
$$
y(x)=x^k\cdot Q_m(x)e^{\mu x},
$$
где $Q_m(x)$ -- многочлен одинаковой с $P_m(x)$ степени $m$, а число $k$ равно кратности корня $\mu$ характеристического уравнения $L(\lambda)=0$ в резонансном случае и $k=0$ в нерезонансном случае.
\end{thm}

\begin{proof}
Если $\mu\neq 0$, то заменой $y=e^{\mu x}\cdot z$ в уравнении \eqref{ch26adiff1.3} всегда можно избавиться от $e^{\mu x}$ в правой части. В самом деле,  используя формулу сдвига, после замены имеем, что
$$
L(D)y=L(D)(e^{\mu x}z)=e^{\mu x}L(D+\mu)z=e^{\mu x}P_m(x).
$$
Отсюда $L(D+\mu)z=P_m(x)$.

Таким образом, доказательство теоремы осталось провести для уравнения вида
\begin{equation}\label{ch26adiff1.5}
L(D)y=P_m(x)
\end{equation}

а) Нерезонансный случай: $L(0)\neq 0$. Пусть
$$
P_m(x)=p_0x^m+p_1x^{m-1}+...+p_m,
$$
$$
Q_m(x)=q_0x^m+q_1x^{m-1}+...+q_m.
$$
Подставляя $P_m(x)$, $Q_m(x)$ в \eqref{ch26adiff1.5} и приравнивая коэффициенты при одинаковых степенях $x$, получаем линейную алгебраическую систему уравнений для  определения неизвестных коэффициентов $q_0,q_1,...,q_m$:
$$
\begin{cases}
a_nq_0=p_0,\\
a_nq_1+a_{n-1}\cdot m q_0=p_1,\\
...........................................\\
a_nq_m+...=p_m.
\end{cases}
$$
Матрица этой линейной системы треугольная с числами $a_n=L(0)\neq 0$ по диагонали, поэтому все коэффициенты $Q_m(x)$ определяются однозначно.

б) Резонансный случай
$$
L(\lambda)=\lambda^k(\lambda^{n-k}+a_1\lambda^{n-k-1}+...+a_{n-k}),\quad 1\leq k \leq n,\quad a_{n-k}\neq 0,\quad k\neq n.
$$
Следовательно,
$$
L(D)=\begin{cases}
D^n+a_1D^{n-1}+...+a_{n-k}D^k, &k\le n,\\
D^n, &k=n.
\end{cases}
$$

В случае $k<n$ замена $D^ky=z$ в уравнении \eqref{ch26adiff1.4} приводит к уравнению
$$
L_1(D)z\equiv (D^{n-k}+...+a_{n-k})z=P_m(x).
$$
Поскольку $L_1(0)=a_{n-k}\neq 0$, то для этого уравнения имеет место нерезонансный случай. Следовательно, существует единственное решение этого уравнения $z=R_m(x)$, где $R_m(x)$ -- некоторый многочлен степени $m$.


Рассмотрим уравнение
$$
D^ky=\begin{cases}
R_m(x), &k<n,\\
P_m(x), &k=n.
\end{cases}
$$

Взяв нулевые начальные условия для этого уравнения
$$
y(0)=y'(0)=...=y^{(k-1)}(0)=0,
$$
получаем единственное решение вида
$$
y(x)=x^k\cdot Q_m(x)
$$
\end{proof}
