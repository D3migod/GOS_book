\chapter{Вычеты. Вычисление интегралов по замкнутому контуру при помощи вычетов.}
\section{Вычеты}

\begin{defn}
Пусть $a \in \bbC$ --- изолированная особая точка регулярной функции $f : \overset{\circ}{B}_\rho(a) \to \bbC, \: \rho > 0$. Пусть $\gamma_r \triangleq \{ z \: \big| \: |z - a| = r\}$ --- положительно ориентированная окружность, причем $0 < r < \rho$. Тогда \textit{вычетом функции $f$ в точке $a$} называется число

\begin{equation} \label{ch36.1eq1}
\res\limits_{a} f \triangleq \frac{1}{2\pi i} \int_{\gamma_r} f(z) \,dz.
\end{equation}

\end{defn}

Отметим, что в формуле $\eqref{ch36.1eq1}$ интеграл не зависит от величины $r\in (0, \rho)$.

Для получения более удобных выражений вычисления вычета функции, представим функцию $f : \overset{\circ}{B}_\rho(a) \to \bbC$ ее рядом Лорана с центром в точке $a$

\begin{equation} \label{ch36.1eq2}
f(z) = \sum_{n = -\infty}^{+\infty} c_n (z - a)^n.
\end{equation}

Так как
$$
c_n = \frac{1}{2\pi i} \int_{\gamma_r} \frac{f(\zeta)}{(\zeta - a)^{n + 1}} \,d\zeta, \quad r \in (0, \rho), \quad n \in \bbZ,
$$

то получаем, что интеграл $\eqref{ch36.1eq1}$ равен коэффициенту $c_{-1}$, т.е.

\begin{equation} \label{ch36.1eq3}
\res\limits_{a} f = c_{-1}.
\end{equation}

Приведем некоторые правила вычисления вычетов.

\begin{lemm} \label{ch36.1lemm1}
Пусть $a$ --- полюс функции $f$ порядка $m \le m_0$. Тогда справедлива формула

\begin{equation} \label{ch36.1eq4}
\res\limits_{a} f = \frac{1}{(m_0 - 1)!} \lim_{z \to a} \frac{\,d^{m_0 -1}}{\,dz^{m_0 - 1}} [(z - a)^{m_0} f(z)].
\end{equation}

\end{lemm}

\begin{proof}
Представим функцию $f$ в виде ряда Лорана $\eqref{ch36.1eq2}$ с центром в полюсе $a$ порядка $m$. Так как число $m_0 \ge m$, то в ряде $\eqref{ch36.1eq2}$ коэффициенты $c_n = 0$ при всех $n < -m_0$. Итак,

\begin{equation} \label{ch36.1eq5}
f(z) = \frac{c_{-m_0}}{(z - a)^{m_0}} + \frac{c_{-m_0 + 1}}{(z - a)^{m_0 - 1}} + \ldots + \frac{c_{-1}}{(z - a)} + c_0 + c_1 (z - a) + \ldots.
\end{equation}

Умножая ряд $\eqref{ch36.1eq5}$ на $(z - a)^{m_0}$, получаем 

\begin{equation} \label{ch36.1eq6}
(z - a)^{m_0} f(z) = c_{-m_0} + c_{-m_0 + 1} (z - a) + \ldots + c_{-1} (z - a)^{m_0 - 1} + \ldots, \quad z \in \overset{\circ}{B}_\rho(a).
\end{equation}

Так как полученный в правой части равенства $\eqref{ch36.1eq6}$ степенной ряд сходится в $B_\rho(a)$, то по теореме \hyperref[ch34.2Thm1]{Абеля} (Билет №34) он сходится абсолютно и равномерно внутри области $B_\rho(a)$. Поэтому по \hyperref[ch35.1Thm5]{теореме \ref{ch35.1Thm5} (Вейерштрасса)} его можно почленно дифференцировать $(m_0 - 1)$ раз, после чего получаем

\begin{equation} \label{ch36.1eq7}
\frac{\,d^{m_0 - 1}}{\,dz^{m_0 - 1}} [(z - a)^{m_0} f(z)] = (m_0 - 1)! c_{-1} + m_0! c_0 (z - a) + \ldots, \quad z \in \overset{\circ}{B}_\rho(a).
\end{equation}

Левая часть равенства $\eqref{ch36.1eq7}$, очевидно, имеет предел при $z \to a$. Поэтому, переходя к пределу, в силу формулы $\eqref{ch36.1eq3}$ получаем формулу $\eqref{ch36.1eq4}$.

\end{proof}

\begin{lemm}
Пусть функция $f : \overset{\circ}{B}_\rho(a) \to \bbC$ представима в виде

$$
f(z) = \frac{P(z)}{Q(z)}, \quad z \in \overset{\circ}{B}_\rho(a),
$$

где функции $P$ и $Q$ регулярны в круге $B_\rho(a)$, причем

\begin{equation} \label{ch36.1eq8}
P(a) \not= 0, \quad Q(a) = 0, \quad Q'(a) \not= 0.
\end{equation}

Тогда спаведлива формула

\begin{equation} \label{ch36.1eq9}
\res\limits_{a} f = \frac{P(a)}{Q'(a)}.
\end{equation}

\end{lemm}

\begin{proof}
В самом деле, в силу условия $\eqref{ch36.1eq8}$ точка $a$ --- полюс 1-го порядка функции $f$ и по формуле  $\eqref{ch36.1eq4}$ получаем

$$
\res_{a} f = \lim_{z \to a} \left[ \frac{P(z)(z - a)}{Q(z)}\right] = \lim_{z \to a} \frac{P(z)}{\frac{Q(z) - Q(a)}{z - a}} = \frac{P(a)}{Q'(a)}.
$$

\end{proof}

\begin{defn} \label{ch36defn2}
Пусть функция $f : \overset{\circ}{B}_{R_0}(\infty) \to \bbC$ регулярна (число $R_0 \ge 0$). Тогда \textit{вычетом функции $f$ в бесконечности} называется число

\begin{equation} \label{ch36.1eq10}
\res_{\infty} f \triangleq \frac{1}{2\pi i} \int_{\gamma_R^{-1}} f(z) \,dz,
\end{equation}

где число $R > R_0$, а окружность $\gamma_R^{-1} = \{ z \: \big| \: |z| = R \}$ ориентирована движением по ходу часовой стрелки (рис. 1)(т. е. отрицательно).
\end{defn}

Аналогично случаю конечной точки оценим $\res_{\infty} f$ через ряд Лорана для функции $f$ в окрестности $\overset{\circ}{B}_{R_0}(\infty)$, учитывая, что его коэффициенты имеют вид

\begin{equation} \label{ch36.1eq11}
c_n = \frac{1}{2\pi i} \int_{\gamma_R} \frac{f(\zeta)}{\zeta^{n + 1}} \,d\zeta, \quad n \in \bbZ,
\end{equation}

где окружность $\gamma_R$ при $R > R_0$ ориентирована движением против хода часовой стрелки. Сравнивая выражения $\eqref{ch36.1eq11}$ и $\eqref{ch36.1eq10}$, убеждаемся в справедливости формулы

\begin{equation} \label{ch36.1eq12}
\res_{\infty} f = -c_{-1},
\end{equation}

где $c_{-1}$ --- коэффициент разложения функции $f$ в ряд Лорана с центром в бесконечности. Здесь появился знак минус за счет различной ориентации окружности $\gamma_R$ в формулах $\eqref{ch36.1eq11}$ и $\eqref{ch36.1eq10}$.

\begin{lemm} \label{ch36.1lemm3}
Пусть $\infty$ --- устранимая точка функции $f$. Тогда $\res\limits_{\infty} f$ можно вычислить по формуле

\begin{equation} \label{ch36.1eq13}
\res_{\infty} f = \lim_{z \to \infty} [z(f(\infty) - f(z))].
\end{equation}

\end{lemm}

\begin{proof}
Из условия леммы следует, что ряд Лорана в некоторой окрестности $\overset{\circ}{B}_{R_0}(\infty)$ имеет вид

$$
f(z) = f(\infty) + \frac{c_{-1}}{z} + \frac{c_{-2}}{z^2} + \ldots,
$$

т.е.

$$
z(f(\infty) - f(z)) = -c_{-1} - \frac{c_{-2}}{z} + o\left(\frac{1}{z}\right),
$$

что в пределе при $z \to \infty$ дает формулу $\eqref{ch36.1eq13}$.	

\end{proof}

\begin{thm} [Коши о вычетах] \label{ch36.1T1}
Пусть дана область $G \subset \overline{\bbC}$ с кусочно-гладкой положительно ориентированной границей $\Gamma$ (см. определения \ref{ch33defn1}, \ref{ch33defn2} из билета №33). Пусть функция $f$ определена и регулярна на $G$ всюду, за исключением конечного числа изолированных особых точек $a_1,a_2,\ldots,a_n \in G$ (при этом имеется в виду, что все $a_k$ различны и если $\infty \in G$, то $\infty = a_n$) и пусть к тому же функция $f$ непрерывно продолжима на границу области $G$. Тогда справедлива формула

\begin{equation} \label{ch36.1eq14}
\int_\Gamma f(z) \,dz = 2 \pi i \sum_{k = 1}^{n} \res_{a_k} f.
\end{equation}

\end{thm}

\begin{proof}
\begin{enumerate}
\item Пусть область $G$ ограничена. Так как число особых точек $a_1,\ldots,a_n \in \bbC$ конечно, то существует число $r > 0$ такое, что $B_r(a_k) \subset G \quad \forall k \in \overline{1,n}$, причем замыкания этих кругов попарно не пересекаются. Определим множество $\widetilde{G} \triangleq G \setminus \left( \bigcup\limits_{k = 1}^{n} \overline{B_r(a_k)}\right)$.

Множество $\widetilde{G}$ тоже является областью с кусочно-гладкой границей $\widetilde{\Gamma} = \Gamma \cup \left( \bigcup\limits_{k = 1}^{n} \gamma_{k}^{-1} \right)$,	где $\gamma_k$ суть окружности $\{ z \: \big| \: |z - a_k| = r \}$, ориентированные движением против хода часовой стрелки, а $\gamma_{k}^{-1}$ --- они же, но ориентированные по ходу часовой стрелки. Получили, что $f$ регулярна на $\widetilde{G}$ и непрерывна на $\overline{\widetilde{G}} = \widetilde{G} \cup \widetilde{\Gamma}$ (см. рис. 2). Тогда по \hyperref[abc28]{обобщенной теореме\ref{abc28} Коши} (Билет №33) получаем

$$
0 = \int_{\widetilde{\Gamma}} f(z) \,dz = \int_{\Gamma} f(z) \,dz - \sum_{k = 1}^{n} \int_{\gamma_k} f(z) \,dz \myeq{\eqref{ch36.1eq1}} \int_{\Gamma} f(z) \,dz - \sum_{k = 1}^{n} 2\pi i \res_{a_k} f,
$$

что и дает формулу $\eqref{ch36.1eq14}$.


\item	Пусть $\infty \in G$. Тогда особые точки $a_1,\ldots,a_{n - 1} \in \bbC$ --- конечны, а $a_n = \infty$. Так как по \hyperref[ch33defn1]{определению \ref{ch33defn1}} (Билет №33) граница $\Gamma$ состоит из ограниченных гладких компонент, то существует число $R > 0$ такое, что для каждого $z \in \Gamma \cup \left( \bigcup\limits_{k = 1}^{n - 1} a_k \right)$ справедливо неравенство $|z| < R$.

Определим $\widetilde{G} = G \cap B_R(0)$. Тогда $\widetilde G$ — ограниченная область с кусочно-гладкой границей $\widetilde{\Gamma} = \Gamma \cup \gamma_R$, где $\gamma_R = \{ z \: \big| \: |z| = R\}$ --- окружность, ориентированная движением против хода часовой стрелки (см. рис. 3). Для регулярной в $\bbC \setminus B_R(0)$ функции $f$ по \hyperref[ch36defn2]{определению \ref{ch36defn2}} справедлива формула

\begin{equation} \label{ch36.1eq15}
\res_{\infty} f = - \frac{1}{2\pi i} \int_{\gamma_R} f(z) \,dz.
\end{equation}


Так как область $\widetilde{G}$ ограничена, то, опираясь на результат п.1, получаем

\begin{equation} \label{ch36.1eq16}
\int_{\widetilde{\Gamma}} f(z) \,dz = 2\pi i \sum_{k = 1}^{n - 1} \res_{a_k} f.
\end{equation}
 
С другой стороны,

$$
\int_{\widetilde{\Gamma}} f(z) \,dz = \int_{\Gamma} f(z) \,dz + \int_{\gamma_R} f(z) \,dz \myeq{\eqref{ch36.1eq15}} \int_{\Gamma} f(z) \,dz - 2\pi i\res_{\infty} f,
$$

откуда и из $\eqref{ch36.1eq16}$ следует $\eqref{ch36.1eq14}$.	

\end{enumerate}
\end{proof}

\begin{cons}
Пусть функция $f$ регулярна во всей комплексной плоскости, за исключением конечного числа изолированных особых точек $a_1,\ldots,a_n \in \overline{\bbC}$. Тогда

\begin{equation} \label{ch36.1eq17}
\sum_{k = 1}^{n} \res_{a_k} f = 0.
\end{equation}
\end{cons}

\begin{proof}
Так как $\infty$, очевидно, является особой точкой данной функции $f$, то без ограничения общности полагаем, что $a_n = \infty$. Рассмотрим $R > 0$ такое, что все $a_k \in B_R(0) \quad \forall k \in \overline{1, n - 1}$. Как обычно, обозначим через $\gamma_R \triangleq \{ z \: \big| \: |z| = R\}$ окружность, ориентированную движением против хода часовой стрелки.

Тогда по \hyperref[ch36.1T1]{теореме \ref{ch36.1T1}} для области $B_R(0)$ получаем

\begin{equation} \label{ch36.1eq18}
\int_{\gamma_R} f(z) \,dz = 2\pi i \sum_{k = 1}^{n - 1} \res_{a_k} f.
\end{equation}

С другой стороны, по \hyperref[ch36defn2]{определению \ref{ch36defn2}},

$$
- \int_{\gamma_R} f(z) \,dz = 2\pi i \res_{\infty} f,
$$

что вместе с $\eqref{ch36.1eq18}$ дает равенство $\eqref{ch36.1eq17}$.
\end{proof}

\section{Вычисление интегралов по замкнутому контуру при помощи вычетов}

{\bf I. Вычисление интегралов вида}

\begin{equation} \label{ch36.1eq19}
J = \int_{-\infty}^{+\infty} R_{n,m} (x) dx,
\end{equation}

где $R_{n,m}(x) = \frac{P_n(x)}{Q_m(x)}$ --- рациональная функция,

$$
P_n(x) = x^n + a_{n - 1} x^{n - 1} + \ldots + a_0, \quad
Q_m(x) = x^m + b_{m - 1} x^{m - 1} + \ldots + b_0,
$$

причем полагаем, что $Q_m(x) \not= 0$ при всех $x \in \bbR^1$.

Известно, что интеграл $J$ $\eqref{ch36.1eq19}$ сходится при условии $m > n + 1$, что считаем выполненным.

Для вычисления интеграла $\eqref{ch36.1eq19}$ определим ориентированный движением против хода часовой стрелки контур $\gamma_R \triangleq [-R, R] \cup C_R$, где $R > R_0 = \max\{ |z_k^{+}| \: \big| \: k \in \overline{1,l} \}$, а $ \{ z_{k}^{+} \}^l_{k = 1}$ --- совокупность всех различных нулей многочлена $Q_m(z)$, лежащих в верхней полуплоскости $\Im z \ge 0$, полуокружность $C_R \triangleq \{ z \: \big| \: |z| = R, \: \Im z \ge 0 \}$. Чтобы воспользоваться \hyperref[ch36.1T1]{теоремой \ref{ch36.1T1} (Коши о вычетах)}, определим интеграл 

$$
J_R = \int_{\gamma_R} R_{n,m}(z)\,dz.
$$

По \hyperref[ch36.1T1]{теореме \ref{ch36.1T1} (Коши о вычетах)}, при каждом $R > R_0$

$$
J_R = 2\pi i\left( \sum_{k = 1}^{l} \res_{z_k^+} R_{n,m}\right).
$$

С другой стороны, имеет место представление интеграла $J_R = J_R^1 + J_R^2$, где 

\begin{equation} \label{ch36.1eq20}
J_R^1 \triangleq \int_{-R}^{+R} R_{n,m}(x)\,dx, \quad J_R^2 \triangleq \int_{C_R} R_{n,m}(z)\,dz.
\end{equation}

Очевидно, что $\lim\limits_{R \to +\infty} J_R^1 = J$. Осталось показать, что $\lim\limits_{R \to +\infty}J_R^2 = 0$, откуда последует формула:

\begin{equation} \label{ch36.1eq21}
\int_{-\infty}^{+\infty}R_{n,m}(x)\,dx = 2\pi i\sum_{k = 1}^{l} \res_{z_k^+} R_{n,m}.
\end{equation}

Докажем необходимое утверждение.

\begin{lemm} \label{ch36.1lemm4}
Пусть $\Phi(z)$ --- непрерывная функция на замкнутом множестве $\{ z \: \big| \: \Im z \ge 0, \: |z| \ge R_0 > 0 \}$. Пусть $C_R \triangleq \{ z \: \big| \: |z| = R, \: \Im z \ge 0 \}, \: R > R_0$, --- семейство полуокружностей в верхней полуплоскости. Обозначим

$$
\epsilon(R) \triangleq \max\{|\Phi(z)| \: \big| \: z\in C_R \} \quad \text{при} \quad R > R_0.
$$

Если $\lim\limits_{R \to +\infty} \epsilon(R)R = 0$, то $\lim\limits_{R \to +\infty} \int_{C_R} \Phi(z)\,dz = 0$.
\end{lemm}

\begin{proof}
Из условий леммы получаем оценки

$$
\left| \int_{C_R} \Phi(z)\,dz \right| \le \int_{C_R} |\Phi(z)||\,dz| \le \epsilon(R) \int_{C_R}|\,dz| = \epsilon(R)\pi R \to 0.
$$

\end{proof}

Применим \hyperref[ch36.1lemm4]{лемму \ref{ch36.1lemm4}} для случая рациональной функции $\Phi(z) = R_{n,m}(z)$ из правого интеграла в $\eqref{ch36.1eq20}$ при сформулированных выше условиях (т.е. при $m > n + 1$).

При достаточно больших $|z| > R_0$ очевидно справедлива оценка $|\Phi(z)| \le 2|z|^{n - m}$, т.е.
$\epsilon(R)R \le 2R^{n - m + 1} \to 0$, откуда следует, что выполнены условия леммы \ref{ch36.1lemm4}, по которой получаем равенство $\lim\limits_{R \to +\infty} J_R^2 = 0$.

Таким образом, формула $\eqref{ch36.1eq21}$ обоснована полностью.

{\bf II. Вычисление интегралов вида}

\begin{equation} \label{ch36.1eq22}
J = \int_{0}^{2\pi} R(\cos \phi, \sin\phi)\,d\phi,
\end{equation}

где $R(x,y) = \frac{P_n(x,y)}{Q_m(x,y)}; \: P_n, Q_m$ --- многочлены переменных $x$ и $y$.

Сделаем замену $z = z(\phi) = e^{i \phi}, \: 0 \le \phi \le 2\pi$. Тогда 

$\cos\phi = \frac{1}{2} (e^{i\phi} + e^{-i\phi}) = \frac{1}{2}\left( z + \frac{1}{z}\right), \: \sin\phi = \frac{1}{2i} \left(z - \frac{1}{z} \right), \,dz = i e^{i \phi} \,d\phi,$ т.е. 
\begin{multline*}
J = \int_0^{2\pi} R\left( \frac{z(\phi)}{2} + \frac{1}{2z(\phi)}, \: \frac{z(\phi)}{2i} - \frac{1}{2iz(\phi)}\right) \cdot \frac{z'(\phi)}{iz(\phi)} \,d\phi =\\= \int_{|z| = 1} R\left( \frac{1}{2} \left( z + \frac{1}{z}\right), \frac{1}{2i}\left( z - \frac{1}{z}\right)\right)\frac{\,dz}{iz} = \int_{|z| = 1} R_1(z)\,dz.
\end{multline*}

В итоге интеграл $\eqref{ch36.1eq22}$ свелся к интегралу по кругу $|z| = 1$ от рациональной функции $R_1(z)$, который легко может быть вычислен с помощью  \hyperref[ch36.1T1]{теоремы \ref{ch36.1T1} (Коши о вычетах)}.

{\bf III. Вычисление интегралов вида}

\begin{equation} \label{ch36.1eq23}
J = \int_{-\infty}^{+\infty} e^{i\alpha x} R(x) \,dx,
\end{equation}

где $R(x) = \frac{P_n(x)}{Q_m(x)}$ --- рациональная функция, причем $m - n \ge 1, \: \alpha$ --- действительное число, $\alpha > 0$ и $Q_m(x) \not= 0$ при всех $x \in \bbR^1$ (т.е. интеграл $\eqref{ch36.1eq23}$ есть преобразование Фурье рациональной функции $R$).

Для получения условий сходимости интеграла $\eqref{ch36.1eq23}$ представим подынтегральную функцию в виде

$$
R(x) e^{i \alpha x} = \frac{e^{i \alpha x}}{x^{m - n}} + O \left( \frac{1}{x^{m - n + 1}} \right) \quad |x| > 1.
$$

Интеграл $\int_1^{+\infty} \frac{e^{i \alpha x}}{x^{m - n}} \,dx$ сходится при $m - n \ge 1$ (по признаку Дирихле: функция $\int_1^x e^{i\alpha t}\,dt$ ограничена по модулю, а функция $\frac{1}{x^{m - n}}$ монотонно стремится к нулю при $z \to +\infty$). Интеграл 
$\int_{1}^{+\infty} O \left( \frac{1}{x^{m - n + 1}} \right)\,dx$ сходится абсолютно, так как по условию $m - n + 1 \ge 2$. Сходимость интеграла $\eqref{ch36.1eq23}$ в $-\infty$ при $m - n \ge 1$ проверяется аналогично.

Рассмотрим, как и в пункте $I$, положительно ориентированный контур $\gamma_R \triangleq [-R,R] \cup C_R$. При достаточно больших $R > R_0 = \max \{ |z_k^+| \: \big| \: k \in \overline{1,l} \}$ получаем

\begin{equation} \label{ch36.1eq24}
\int_{\gamma_R} e^{i \alpha z}R(z) \,dz = 2\pi i \sum_{k = 1}^l \res_{z_k^+} (e^{i\alpha z}R(z)),
\end{equation}

где через $\{ z_k^+ \}$ обозначены все различные нули многочлена $Q_m(z)$ (знаменателя функции $R(z)$), лежащие в верхней полуплоскости.

С другой стороны, справедливо представление интеграла 

\begin{equation} \label{ch36.1eq25}
\int_{\gamma_R} e^{i \alpha z}R(z)\,dz = \int_{-R}^{+R} e^{i \alpha x}R(x)\,dx + \int_{C_R} e^{i \alpha z} R(z)\,dz.
\end{equation}

Первое слагаемое справа в $\eqref{ch36.1eq25}$, очевидно, стремится к искомому значению $J$ интеграла $\eqref{ch36.1eq23}$ при $R \to +\infty$. Необходимо исследовать второе слагаемое в $\eqref{ch36.1eq25}$.

\begin{lemm} [Жордана] \label{ch36.1lemm5}
Пусть $\Phi(z)$ --- непрерывная функция на замкнутом множестве $\{ z \: \big| \: \Im z \ge 0, |z| \ge R_0 > 0 \}$. Пусть число $\alpha > 0$ и $C_R \triangleq \{ z \: \big| \: |z| = R, \: \Im z \ge 0 \}, \: R > R_0$ --- семейство полуокружностей в верхней полуплоскости. Обозначим

$$
\epsilon(R) \triangleq \max \{ |\Phi(z)| \: \big| \: z \in C_R \} \quad R > R_0.
$$

Если $\lim\limits_{R \to +\infty} \epsilon(R) = 0$, то $\lim\limits_{R \to +\infty} \int_{C_R} e^{i \alpha z} \Phi(z) \,dz = 0$.

\end{lemm}

\begin{proof}
Пусть $z \in C_R$, тогда $z = Re^{i\phi} = Rcos\phi + iRsin\phi, \: 0 \le \phi \le \pi$. Поэтому при $z \in C_R$

$$
|e^{i \alpha z}| = |e^{i \alpha(x + iy)}| = e^{-\alpha y} = e^{- \alpha R \sin\phi}.
$$

Воспользуемся для оценки неравенством

\begin{equation} \label{ch36.1eq26}
\sin \phi \ge \frac{2}{\pi}\phi \quad \text{} \quad \phi \in \left[ 0, \frac{\pi}{2}\right].
\end{equation}

Получаем

\begin{multline*}
\left| \int_{C_R} e^{i \alpha z} \Phi(z)\,dz \right| \le \int_{C_R} |\Phi(z)|e^{-\alpha R \sin \phi}|\,dz| \le \epsilon(R)R \int_0^\pi e^{-\alpha R \sin\phi} \,d\phi =\\= 2\epsilon(R)R \int_0^{\frac{\pi}{2}} e^{-\alpha R \sin\phi} \,d\phi \le 2\epsilon(R)R \int_0^{\frac{\pi}{2}} e^{-\alpha R 2\phi / \pi} \,d\phi \le \frac{\pi}{\alpha} \epsilon(R),\\
\end{multline*}

т.е. справедливо заключение леммы.
\end{proof}

Возвращаясь к формуле $\eqref{ch36.1eq25}$, покажем с помощью \hyperref[ch36.1lemm5]{леммы \ref{ch36.1lemm5} Жордана}, что второй интеграл в $\eqref{ch36.1eq25}$ стремится к нулю при $R \to \infty$. При достаточно больших $R > R_0$ в силу условий сходимости $m - n \ge 1$ получаем $\epsilon(R) \le \frac{2}{R^{m - n}} \le \frac{2}{R} \to 0$. В силу леммы  \ref{ch36.1lemm5} имеет место равенство $\lim\limits_{R \to +\infty} \int_{\gamma_R} e^{i \alpha z}R(z) \,dz = J$, а потому справедлива формула

\begin{equation} \label{ch36.1eq27}
\int_{-\infty}^{+\infty} e^{i \alpha x}R(x) \,dx = 2\pi i \cdot \sum_{k = 1}^l \res_{z_k^+} (e^{i \alpha z}R(z)).
\end{equation}

\begin{cons} \label{ch36.1cons2}
Интегралы вида

\begin{equation} \label{ch36.1eq28}
J_1 = \int_{-\infty}^{+\infty} \cos\alpha x \cdot R(x) \,dx \quad \text{и} \quad J_2 = \int_{-\infty}^{+\infty} \sin \alpha x \cdot R(x) \,dx,
\end{equation}

где $R(x)$ --- рациональная функция, сводятся к интегралу вида $\eqref{ch36.1eq23}$, т.е.

$$
J_1 = \Re \int_{-\infty}^{+\infty} e^{i \alpha x} R(x) \,dx, \quad J_2 = \Im \int_{-\infty}^{+\infty} e^{i \alpha x}R(x) \,dx.
$$
\end{cons}
