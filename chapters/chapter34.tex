\chapter{Интегральная формула Коши. Разложение функции, регулярной в окрестности точки, в ряд Тейлора.}
\section{Интегральная формула Коши}

Получим представление функций, регулярных в ограниченной области, при помощи интеграла по границе этой области. С помощью этого представления покажем, что всякая регулярная в области функция бесконечно дифференцируема.


\begin{exmpl}
\label{exmpl1}
Вычислить интеграл 
$$
I_k = \int_{\gamma_r} (z-a)^k\,dz, \quad k \in \bbZ, \; a\in \bbC
$$
где контур $\gamma_r$ есть окружность $\{ z \bigl|\bigr. |z - a| = r>0 \}$,ориентированная движением против хода часовой стрелки.
\end{exmpl}
\begin{solution}
СЮДА РЕШЕНИЕ СУКА ЗАПИСАЛ.
В Итоге
\begin{enumerate}
\item
При $k = -1$ получаем
\item
При $k \ne -1$ получаем
\end{enumerate}
\end{solution}

\begin{thm} \label{T1}
 Пусть $G$ --- ограниченная область в $\bbC$ с кусочно-гладкой положительно ориентированной границей $\Gamma$. Пусть функция $f\colon \overline{G}\to \bbC$ регулярна на $G$ и непрерывна на $G=G\cup\Gamma$. Тогда для любой точки $z\in G$ справедлива интегральная формула Коши вида
 \begin{equation} \label{1}
 f(z) = \frac{1}{2\pi i}\int_\Gamma \frac{f(\zeta)}{\zeta - z}\,d\zeta.
 \end{equation}
\end{thm}
\begin{proof}
Фиксируем произвольную точку $z \in G$. Функция $\frac{f(\zeta)}{\zeta - z}$ регулярна по переменному $\zeta$ в области $G \setminus \{z\}$. Выберем число $r_0 > 0$ такое, что выполнено включение $\overline{B_{r_0}(z)}\subset G$. 

Обозначим через $\gamma \triangleq \{ \zeta \bigl| |\zeta - z| = r \}$ окружность радиуса $r \in (0, r_0)$ ориентированную против хода часовой стрелки. Обозначим множества $G_r \triangleq G \setminus \overline{B_r(z)}$ и $\Gamma_r \triangleq \Gamma \cup \gamma_r^{-1}$. Очевидно, что множество $G_r$ есть область с кусочно-гладкой положительно ориентированной границей $\Gamma_r$ (см. рис. 1). 

По теореме 3 из §7 получаем

 \begin{equation} \label{2}
 0 = \int_{\Gamma_r} \frac{f(\zeta)}{\zeta - z}\,d\zeta = \int_\Gamma \frac{f(\zeta)}{\zeta - z}\,d\zeta - \int_{\gamma_r} \frac{f(\zeta)}{\zeta - z}\,d\zeta.
 \end{equation}
Итак,

\begin{equation} \label{3}
J \triangleq \frac{1}{2\pi i} \int_\Gamma \frac{f(\zeta)}{\zeta - z}\,d\zeta \myeq{\eqref{2}} \frac{1}{2\pi i} \int_{\gamma_r} \frac{f(\zeta)}{\zeta - z}\,d\zeta,  \quad  \forall r: 0 < r < r_0
\end{equation}

Как показано в примере \ref{exmpl1}, справедливо равенство $1 = \frac{1}{2\pi i} \int_{\gamma_r} \frac{1}{\zeta - z}\,d\zeta$, откуда 

$$J - f(z) = \frac{1}{2\pi i} \int_{\gamma_r} \frac{f(\zeta) - f(z)}{\zeta - z}\,d\zeta, \quad  \forall r \in (0,r_0)$$

Так как $f(\zeta)$ непрерывна в точке $z \in G$, то для каждого $\epsilon > 0$ существует $\delta(\epsilon) \in (0, r_0)$ такое, что $\forall \zeta : |\zeta - z| < \delta(\epsilon)$ следует $|f(\zeta) - f(z)| < \epsilon$. Поэтому выбирая $r \in (0, \delta(\epsilon))$, получаем

\begin{equation} \label{4}
|J - f(z)| \le \frac{1}{2\pi} \int_{\gamma_r} \frac{|f(\zeta) - f(z)|}{|\zeta - z|}|\,d\zeta| \le 
\frac{\epsilon}{2 \pi r} \int_{\gamma_r} |\,d\zeta| = \epsilon.
\end{equation}

Так как $\epsilon > 0$ произвольное число, то из (3), (4) следует $J = f(z)$, т.е. формула $\eqref{1}$.
\end{proof}

\begin{defn}
Пусть $\gamma$ — кусочно-гладкий контур в $\bbC$ и пусть $\omega = q(z)$ — непрерывная на $\gamma$ функция. Тогда интеграл вида

\begin{equation} \label{5}
I(z) \triangleq \frac{1}{2 \pi i} \int_\gamma \frac{q(\zeta)}{\zeta - z}\,d\zeta, \quad z \notin \gamma
\end{equation}
называется интегралом типа Коши по контуру $\gamma$ от функции $q$.
\end{defn}

\begin{thm} \label{T2}
При сформулированных в определении 1 условиях функция $I:\bbC \setminus \gamma \to \bbC$ из $\eqref{5}$ определена и дифференцируема бесконечное число раз, причем для производных справедлива формула

\begin{equation} \label{6}
I^{(n)}(z) = \frac{n!}{2\pi i} \int_\gamma \frac{q(\zeta)}{(\zeta - z)^{n+1}}\,d\zeta, \quad n \in \bbN
\end{equation}
\end{thm}
\begin{proof}
\begin{enumerate}
\item 
Докажем формулу $\eqref{6}$ при $n = 1$. Так как функция $q(\zeta)$ непрерывна на контуре $\gamma$, то существует число $M < + \infty$ такое, что $|q(\zeta)| \le M$ при $\zeta \in \gamma$.

Фиксируем точку $z \notin \gamma$. Пусть $d \triangleq dist(z, \gamma)$. Очевидно, что $d > 0$. Выберем число $r \in (0, \frac{d}{2})$ и возьмем произвольное число $\Delta z \in \bbC$ так, чтобы $0 < |\Delta z| < r$. Тогда для $\forall \zeta \in \gamma$ получаем

\begin{equation} \label{7}
|\zeta - (z + \Delta z)| \ge |\zeta - z| - |\Delta z| \ge d - \frac{d}{2} = \frac{d}{2}
\end{equation}

Оценим выражение

\begin{equation} \label{8}
\frac{I(z + \Delta z) - I(z)}{\Delta z} - \frac{1}{2 \pi i} \int_\gamma \frac{q(\zeta)}{(\zeta - z)^2}\,d\zeta = \frac{1}{2 \pi i} \int_\gamma q(\zeta) \left[ \left( \frac{1}{\zeta - z - \Delta z} - \frac{1}{\zeta - z} \right) \frac{1}{\Delta z} - \frac{1}{(\zeta - z)^2} \right] \,d\zeta.
\end{equation}

Упростим выражение в прмых скобках под интегралом $\eqref{8}$:

$$
[\ldots] = \frac{\zeta - z - (\zeta - z - \Delta z)}{(\zeta - z)(\zeta - z - \Delta z)} \cdot \frac{1}{\Delta z} - \frac{1}{(\zeta - z)^2} = \frac{\Delta z}{(\zeta - z)^2 (\zeta - z - \Delta z)} .
$$

Поэтому для $\eqref{8}$ получаем оценку 

$$
\left| \frac{\Delta I}{\Delta z} - \frac{1}{2\pi i} \int_\gamma \frac{q(\zeta)}{(\zeta - z)^2}\,d\zeta \right| 
\le \frac{1}{2\pi} \int_\gamma \frac{|q(\zeta)||\Delta z||\,d\zeta|}{|\zeta - z|^2|\zeta - z - \Delta z|}
\le \frac{|\Delta z|}{\pi d^3} \int_\gamma |q(\zeta)||\,d\zeta| 
\le \frac{|\Delta z| \cdot M}{\pi d^3} \int_\gamma |\,d\zeta| \to 0, \quad \Delta z \to 0.
$$

Таким образом, в пределе получаем равенство

\begin{equation} \label{9}
I'(z) = \frac{1}{2\pi i} \int_\gamma \frac{q(\zeta)}{(\zeta - z)^2} \,d\zeta.
\end{equation}

\item 
Общий случай $n$-й производной получается аналогично первому случаю из формулы $\eqref{6}$ для $(n - 1)$-й производной и воспользовавшись равенством

$$
(\zeta - z - \Delta z)^n = (\zeta - z)^2 - n \Delta z (\zeta - z)^{n - 1} + O(\Delta z^2),
$$

которое легко проверяется, например, методом математической индукции.
\end{enumerate}
\end{proof}

\begin{thm} \label{T3}

Пусть функция $f : G \to \bbC$ регулярна в области $G \subset \bbC$. Тогда эта функция имеет в $G$ производные всех порядков, т.е. является бесконечно дифференцируемой функцией в области $G$.

\end{thm}
\begin{proof}
Фиксируем произвольную точку $z_0 \in G$, тогда существует число $r_0 > 0$ такое, что $\overline{B_{r_0}(z_0)}\subset G$. Пусть окружность $\gamma_{r_0} \triangleq \{ z \: \bigl| \: |z - z_0| = r_0 \}$ ориентирована положительно относительно внутренности круга (т.е. движением против хода часовой стрелки). Тогда по теореме 1 справедлива интегральная формула Коши

\begin{equation} \label{10}
f(z) = \frac{1}{2\pi i} \int_{\gamma_{r_0}} \frac{f(\zeta)}{\zeta - z} \,d\zeta, \quad \forall z \in B_{r_0}(z_0).
\end{equation}

Так как формула $\eqref{10}$ функция $\zeta \to f(\zeta)$ непрерывна на $\zeta_{r_0}$, то интеграл в $\eqref{10}$ есть интеграл типа Коши, и по теореме 2 он бесконечно дифференцируем в круге $B_{r_0}(z_0)$, т.е. в силу равенства $\eqref{10}$ функция $f$ бесконечно дифференцируема в этом круге $B_{r_0}(z_0)$, при этом из $\eqref{6}$ следует формула:

\begin{equation} \label{11}
f^{n}(z) = \frac{n!}{2\pi i} \int_{\gamma_{r_0}} \frac{f(\zeta)}{(\zeta - z)^{n+1}}\,d\zeta, \quad \forall z \in B_{r_0}(z_0).
\end{equation}

Так как точка $z_0 \in G$ была произвольной, то функция $f$ бесконечно дифференцируема во всей области G.
\end{proof}
\section{Разложение функции, регулярной в окрестности точки, в ряд Тейлора}