\chapter[\texorpdfstring{Достаточные условия дифференцируемости функции не\-скольких переменных.}{}]{Достаточные условия дифференцируемости функции нескольких переменных.}
\section[\texorpdfstring{Достаточные условия дифференцируемости функции \\несколь\-ких переменных.}{}]{Достаточные условия дифференцируемости функции нескольких переменных}

\begin{defn}
Функция $f$ называется дифференцируемой в точке $x^0 \in \bbR^n$, если она определена в некоторой окрестности точки $x^0$ и $\exists$ вектор $A = \{ A_1, A_2, \ldots , A_n\} \in \bbR^n$, такой что 

\begin{equation} \label{ch8eq1}
f(x) - f(x^0) = (A, x - x^0) + o(|x - x^0|), \; x \to x^0, 
\end{equation}
где $(A, x - x^0) = A_1(x_1 - x^0_1) + \ldots + A_n(x_n - x^0_n); \; o(|x - x^0|) = \alpha(x)|x - x^0|,$ где $\alpha(x)$ определена в $O(x^0), \; \lim\limits_{x \to x^0} \alpha(x) = 0$.

При этом вектор $A \in \bbR^n$ называется градиентом функции $f$ в точке $x^0$. (Градиент $f$ в точке $x^0$ определен, если функция $f$ дифференцируема в точке $x^0$). Обозначается $\grad f(x^0), \nabla f(x^0)$.

Функция $f$ дифференцируема в точке $x^0 \in \bbR^n \; \Leftrightarrow \; f$ определена в окрестноти $O(x^0)$ и $\exists A \in \bbR^n$, такое что 

$$
f(x) = f(x^0) + (A, x - x^0) + \alpha(x) \cdot |x - x^0|, \; \textit{где} \; \alpha - \textit{бмф при } x \to x^0 
$$ 

$\Leftrightarrow \;$
\begin{equation} \label{ch8eq1'}
\exists \lim\limits_{x \to x^0} \frac{f(x) - f(x^0) - (A, x - x^0)}{|x - x^0|} = 0
\tag{\ref{ch8eq1}$'$}
\end{equation}

\end{defn}

\begin{defn}
Если функция $f$ дифферецируема в точке $x^0 \in \bbR^n$, то дифференциалом функции $f$ в точке $x^0$ называется функция аргумента $h \in \bbR^n$, линейная  по $h$ и заданная равенством

$$
df(x^0, h) = (\grad f(x^0), h) = A_1 h_1 + \ldots + A_n h_n.
$$

Если функция $f: G \to \bbR^1 (G \subset \bbR^n)$, дифференцируема на множестве $D \subset G$, то дифференциалом функции $f$ называется функция $df(x,h)$, где $(x,h) \in D \times \bbR^n$; $df(x,h) = (\grad f(x), h)$
\end{defn}

\begin{defn}
Производной функции $f$, определенной в окрестности $O(x^0), \: x^0 \in \bbR^n$ по направлению $\vv{l} \in \bbR^n \; |l| = 1$, называется предел
$$
\frac{\partial f}{\partial l}(x^0) = \lim\limits_{t \to 0} \frac{f(x^0 + tl) - f(x^0)}{t}
$$
\end{defn}

\begin{defn}
Частной производной функции $f$, определенной в окрестности точки $x^0 \in \bbR^n$ по переменной $x_j, \: j \in \overline{1,n}$ в точке $x^0 \in \bbR^n$, называется производная функции $f$ в точке $x^0$ по направлению $\vv{l}_j = (0, \ldots, l_j = 1, \ldots, 0)$. Обозначается $\frac{\partial f}{\partial x_j}(x^0)$

$$
\frac{\partial f}{\partial x_j}(x^0) = \lim\limits_{t \to 0} \frac{f(x^0_1, \ldots, x^0_{j - 1}, x^0_j + t, x^0_{j + 1}, \ldots, x^0_n) - f(x^0)}{t}  
$$
\end{defn}


\begin{thm}[Достаточное условие дифференцируемости] \label{ch8th1}
Если функция $f$ имеет в некоторой окрестности $O(x^0, y^0)$ частные производные $f'_x(x,y), \: f'_y(x,y), \: (x,y) \in O(x^0, y^0)$, которые непрерывны в точке $(x^0, y^0)$, то функция $f$ дифференцируема в точке $x^0, y^0$.
\end{thm}

\begin{proof}
Пусть $(x,y) \in O_\delta(x^0, y^0)$, $|x - x^0|^2 + |y - y^0|^2 < \delta^2 \; \Rightarrow \; |y - y^0|^2 < \delta^2 \; \Rightarrow \; (x^0, y) \in O_\delta(x^0, y^0) \; \Rightarrow \; $
$$
f(x,y) - f(x^0, y^0) = f(x,y) - f(x^0, y) + f(x^0, y) - f(x^0, y^0),
$$
$ \quad f(x,y) = \phi(\xi), \xi \in [x^0, x] \quad \Rightarrow$ согласно условию Теоремы фунция $f(\xi, y)$ имеет частную производную $\frac{\partial f}{\partial x}(\xi, y)$ в любой точке $(\xi, y)$  при $\xi \in [x^0, x]$, т.к. тогда $(\xi, y) \in O(x^0, y^0) \quad \Rightarrow$ функция $\phi$ дифференцируема на отрезке $[x^0, x] \quad \Rightarrow$ по теореме Лагранжа $\exists \xi_1 \in (x^0, x), \: \phi(x) - \phi(x^0) = \phi'(\xi_1)(x - x^0) \quad \Rightarrow \quad \exists \xi_1 \in (x^0, x)$

$f(x,y) - f(x^0, y) = \frac{\partial f}{\partial x}(\xi, y)(x - x^0)$

Аналогично, $\exists \eta_1 \in (y, y^0)$:
$$
f(x^0, y) - f(x^0, y^0) = \frac{\partial f}{\partial y}(x^0, \eta_1)(y - y^0)
$$ 

$\forall (x,y) \in O_\delta(x^0, y^0) \quad \exists \xi_1 \in (x^0, x), \: \exists \eta_1 \in (y^0, y):$

$$
f(x,y) - f(x^0, y^0) = f'_x(\xi, y) \Delta x + f'_y(x, \eta) \Delta y
$$

Т.к. частные производные непрерывны в точке $(x^0, y^0)$, то

$$
\lim\limits_{\substack{x \to x^0 \\ y \to y^0}} f'_x(\xi_1, y) = f'_x(x^0, y^0) \: \Rightarrow \: \alpha \: = \: f'_x(\xi, y) - f'_x(x^0, y^0) \; - \; \textit{бмф при } \: (x,y) \to (x^0, y^0)
$$

$$
\lim\limits_{\substack{x \to x^0 \\ y \to y^0}} f'_y(x, \eta_1) = f'_y(x^0, y^0) \: \Rightarrow \: \beta \: = \: f'_y(x, \eta_1) - f'_y(x^0, y^0) \; - \; \textit{бмф при } \: (x,y) \to (x^0, y^0)
$$

$\Rightarrow \; f(x,y) - f(x^0, y^0) = f'_x(x^0, y^0) \Delta x + f'_y(x^0, y^0) \Delta y + \alpha \Delta x + \beta \Delta y$.

$\alpha \Delta x + \beta \Delta y = \frac{\alpha \Delta x + \beta \Delta y}{\sqrt{\Delta x^2 + \Delta y^2}} \cdot \sqrt{\Delta x^2 + \Delta y^2} = o(\sqrt{\Delta x^2 + \Delta y^2})$

$$
\frac{\Delta x}{\sqrt{\Delta x^2 + \Delta y^2}}, \; \frac{\Delta y}{\sqrt{\Delta x^2 + \Delta y^2}} - \textit{ограниченые},
$$

$$
\gamma = \alpha \frac{\Delta x}{\sqrt{\Delta x^2 + \Delta y^2}} + \beta \frac{\Delta y}{\sqrt{\Delta x^2 + \Delta y^2}} - \textit{бмф}.
$$

$\Rightarrow \; f$ дифференцируема в точке $(x^0, y^0)$.
\end{proof}

\begin{cons}
Если фунция $f$ определена на области $G \subset \bbR^n$ и имеет на этой области частные производные 1-го порядка, которые непрерывны на области $G$, то функция $f$ дифференцируема в каждой точке области $G$.
\end{cons}