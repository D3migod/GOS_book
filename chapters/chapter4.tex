\chapter{Теоремы о среднем Ролля, Лагранжа и Коши для дифференцируемых функций.}


\section{Теоремы о среднем Ролля, Лагранжа и Коши для дифференцируемых функций}

\subsection{Определение производной}
\textbf{CМОТРИ В БИЛЕТЕ 6 пока что. + странно что материал страниц 116 нигде не пригодился, хотя это фундаментальные понятия и леммы}

\subsection{Теоремы о среднем для дифференцируемых функций}
Пусть задана функция $f$, и пусть $x_0 \in D_f$, где $D_f$ — область определения функции $f$.
\begin{defn}
Точка $x_0 \in D_f$ называется \textit{точкой локальною максимума (минимума) функции} $f$, если существует окрестность $O(x_0)$ точки $x_0$ такая, что для любого $x \in O(x_0)\cap D_f$ выполняется неравенство $f(x) \le f(x_0)$ (соотв., $f(x) \ge f(x_0)$).
\end{defn}
\begin{defn} Точки максимума и минимума функции называются ее \textit{точками экстремума}, а ее значения в этих точках — \textit{экстремальными значениями} (соотв., \textit{локальными максимумами} или \textit{локальными минимумами}).
\end{defn}
\begin{thm}(Ферма)\label{ch4n1} 
Если функция дифференцируема в точке $x_0$ и $x_0$ является ее точкой экстремума, то $f'(x_0) = 0$.
\end{thm}
\begin{proof}
Прежде всего заметим, что функция $f$ определена в некоторой окрестности точки $x_0$, так как она дифференцируема в точке $x_0$. Пусть, например, $x_0$ — точка максимума функции $f$. Тогда существует окрестность $O(x_0)$ точки $x_0$ такая, что 
$$
f(x) \le f(x_0) \fa x \in O(x_0),
$$
и поэтому если $х \in O(x_0)$ и $x < x_0$ (соотв., $x > x_0$), то
$$
\frac{f(x)-f(x_0)}{x-x_0}\ge0\quad (\text{соотв.} \le 0).
$$
Следовательно,
$$
f'_{-}(x_0)\ge 0, \quad f'_{+}(x_0)\le 0
$$
а так как эти односторонние производные равны производной $f'(x_0)$,
то $f'(х_0) = 0$. 

Случай, когда $x_0$ --- точка минимума, рассматривается аналогично. 

\noindent 
Теорема доказана.
\end{proof}

Теорема Ферма имеет простую геометрическую интерпретацию:	если функция $у = f(х)$ дифференцируема в точке $x_0$ и $x_0$ — ее точка экстремума, то касательная к графику этой функции в точке $x_0$ параллельна оси $Ох$. 

\begin{defn}
Функция $f(x)$ называется \textit{дифференцируемой на интервале $(a;b)$}, если она определена на $(a;b)$ и в каждой его точке имеет конечную производную.

\end{defn}
\begin{thm} (Ролля) Если функция $f(x)$ непрерывна на отрезке $[a; b]$, дифференцируема на интервале $(a; b)$ и $f(a) = f(b)$, то существует $\xi \in (a; b)$ такое, что $f'(\xi) = 0$.
\end{thm}
\begin{proof}
Функция $f(x)$ на отрезке $[a;b]$ принимает и наибольшее и наименьшее значения, а так как $f(a) = f(b)$, то одно из них она принимает в некоторой точке $\xi \in (a; b)$. Тогда из теоремы \hyperref[ch4n1]{Ферма} следует, что $f'(\xi) = 0$. 
\noindent Теорема доказана.
\end{proof}

\begin{thm} Если функция $f(x)$ непрерывна на отрезке $[a; b]$ и дифференцируема на интервале $(a; b)$, то существует $\xi \in (a; b)$ такое, что
\begin{equation}\label{ch4e1}
\frac{f(b)-f(a)}{b-a}=f'(\xi)
\end{equation}
\end{thm}

\begin{proof} Рассмотрим функцию $F(x) = f(x) - \lambda х$ и найдем $\lambda$ из условия $F(a) = F(b)$. Тогда
\begin{equation}\label{ch4e2}
\lambda=\frac{f(b)-f(a)}{b-a}
\end{equation}
Функция $F(x)$ при таком $\lambda$ удовлетворяет всем условиям теоремы Ролля, поэтому существует $\xi \in (a; b)$ такое, что $F'(\xi) = 0$. А так как $F'(x) = f'(x) - \lambda$, то $f'(\xi) = \lambda$, т.е. выполняется равенство \eqref{ch4n1}. 
\noindent Теорема доказана.
\end{proof}
Заметим, что коэффициент $\lambda$, определяемый по формуле \eqref{ch4e2}, равен угловому коэффициенту хорды, проходящей через точки $A(a;f(a))$ и $B(b;f(b))$. Следовательно, теорема Лагранжа утверждает существование точки, в которой касательная к графику функции $f$ параллельна этой хорде.
\begin{cons} 
Если функция $f(x)$ непрерывна в некоторой окрестности $O(х_0)$ точки $x_0$ и дифференцируема в проколотой
окрестности $\dnei{x_0}$, то для любого $х \in \dnei{x_0}$ существует $\xi$ лежащее строго между $х$ и $х_0$ и такое, что
\begin{equation}\label{ch4e3}
f(x)-f(x_0)=f'(\xi)(x-x_0).
\end{equation}
\end{cons}

Формула \eqref{ch4e3} называется \textit{формулой Лагранжа} для конечных приращений или \textit{формулой конечных приращений}.
Это следствие можно интерпретировать несколько иначе:

\begin{cons}
Если функция $f$ непрерывна в $O(x_0)$ и дифференцируема в $\dnei(x_0)$, то $\forall x \in \dnei{х_0} \ex \Theta \in (0;1)$:
$$
f(x) - f(x_0) = f'(x_0 + \Theta\Delta x)\Delta x,
$$
где $\Delta x=x-x_0$.
\end{cons}

\begin{cons}
Если функция $f$ непрерывна на отрезке $[a; b]$, дифференцируема на интервале $(a;b)$ и $f'(x) = 0 \fa x \in (a;b)$, то функция $f$ постоянна на отрезке $[a; b]$.	
\end{cons}

\begin{defn}Функции называется \textit{кусочно-дифференцируемой} на некотором промежутке, если она всюду, кроме конечного числа точек, имеет конечную производную.
\end{defn}

\begin{cons} Если функция непрерывна на некотором конечном или бесконечном промежутке и всюду, кроме конечного числа точек, имеет производную, равную нулю, то зта функция постоянна на рассматриваемом промежутке.
\end{cons}

\begin{thm} (Коши) Если функции $f(x)$ и $g(x)$ непрерывны на отрезке $[a; b]$, дифференцируемы на интервале $(a; b)$ и $g'(x) \ne 0$ на $(a; b)$, то существует $\xi \in (a; b)$ такое, что
\begin{equation}
\label{ch4e4}
\frac{f(b)-f(a)}{g(b)-g(a)}=\frac{f'(\xi)}{g'(\xi)}.
\end{equation}
\end{thm}

\begin{proof} Так как $g'(x) \ne 0$ на $(a,b)$, то $g(b)\ne g(а)$ (по теореме Ролля). Рассмотрим функцию $F(x) =f(x) - \lambda g(x)$ и найдем $\lambda$ из условия $F(a) = F(b)$. Тогда
$$
\lambda = \frac{f(b)-f(a)}{g(b)-g(a)}
$$
Функция $F(х)$ при таком $\lambda$ удовлетворяет всем условиям теоремы Ролля, поэтому существует $\xi \in (a; b)$ такое, что $F'(xi) = 0$. А так как $F(x) = f'(х) - \lambda g'(х)$, то $f'(\xi) = \lambda g'(\xi)$, т.е. выполняется равенство \eqref{ch4e4}.

\noindent Теорема докапана.
\end{proof}

\begin{cons} Если функции $f(x)$ и $g(x)$ непрерывны в некоторой окрестности $O(x_0)$ точки $x_0$ и дифференцируемы в проколотой
окрестности $\dnei{x_0}$, причем $g'(x) \ne 0$ в  $\dnei{x_0}$, то  для любого $x \in  \dnei{x_0}$ существует $\xi$, лежащее строго между $x$ и $х_0$ такое, что
$$
\frac{f(x)-f(x_0)}{g(x)-g(x_0)}=\frac{f'(\xi)}{g'(\xi)}.
$$
\end{cons}