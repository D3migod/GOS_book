\chapter[Теорема о равномерной непрерывности функции, непрерывной на компакте.]{Теорема о равномерной непрерывности функции, непрерывной на компакте.}
\section{Теорема о равномерной непрерывности функции, непрерывной на компакте}

\begin{defn}
Множество $G$ точек из $\bbR ^n$ называется \textit{ограниченным}, если существует число $r\ge 0$ такое, что
$$
|OM|\le r \quad \forall M\in G.
$$
(Здесь $O$ --- точка с координатами $(0, 0, \ldots , 0).$)
\end{defn}

\begin{defn}
Точка $M_0$ называется \textit{точкой прикосновения множества $G$}, если в любой ее окрестности содержится хотя бы одна точка из $G$.
\end{defn}

\begin{defn}
Множество всех точек прикосновения множества $G$ называется \textit{замыканием} множества $G$ и обозначается $\overline{G}$.
\end{defn}

\begin{defn}
Множество называется \textit{замкнутым}, если оно совпадает со своим замыканием.
\end{defn}

Для любого множества $G$ все его точки и все его предельные точки являются точками прикосновения, и других точек прикосновения нет.

\begin{defn}
Всякое ограниченное замкнутое множество $G \subset \bbR^n$ будем называть компактным множеством в $\bbR^n$ (компактом в $\bbR^n$ ).
\end{defn}


\subsection{Равномерно непрерывные функции и отображения}

\begin{defn}
Функция $f(M),\;M\in G$, называется \textit{непрерывной на множестве $G$}, если она непрерывна в каждой его точке, т.е. если выполняется условие:
\begin{equation}\label{yaa45e1}
\forall M_0\in G\quad \forall\epsilon>0 \quad \exists\delta>0:\;\forall M\in G,\quad |MM_0|<\delta:\;|f(M)-f(M_0)|<\epsilon.
\end{equation}
\end{defn}
Заметим, что здесь $\delta$ зависит как от $\epsilon$, так и от $M_0$.

\begin{defn}
Функция $f(M),\; M\in G$, называется \textit{равномерно непрерывной на множестве G}, если выполняется условие:
\begin{equation}\label{yaa45e2}
\forall\epsilon>0\;\exists\delta>0:\quad\forall M, M'\in G,\; |MM'|<\delta\; :\; |f(M)-f(M')|<\epsilon.
\end{equation}
\end{defn}
Заметим, что здесь $\delta$ зависит только от $\epsilon$.

Очевидно, если выполнено условие \eqref{yaa45e2}, то и выполнено условие \eqref{yaa45e1}, т.е. если функция равномерно непрерывна в любой точке этого множества. Как показывают примеры, обратное утверждение является неверным.

\begin{thm}
Если функция $f(M)$ определена и непрерывна на ограниченном замкнутом множестве $G\subset\bbR^n$ (т.е. $G$ --- компакт в $\bbR^n$), то она равномерно непрерывна на $G$.
\end{thm}

\begin{proof}
Доказывать будем методом от противного. Предположим, что функция $f(M)$ не является равномерно непрерывной на $G$. Тогда
$$
\exists\epsilon_0>0:\quad\forall\delta>0\quad\exists M,M'\in G:\quad |MM'|<\delta, \quad |f(M)-f(M')|\ge \epsilon_0
$$
Через $M_k$ и $M_k'$ обозначим точки из этого условия, которые соотвествуют $\delta=1/k$, т.е. $M_k$ и $M_k'$ принадлежат множеству $G$ и такие, что 
\begin{equation}\label{yaa45e4}
|M_kM_k'|<\frac{1}{k},\quad|f(M_k)-f(M_k')|\ge \epsilon_0.
\end{equation}

Так как множество $G$ ограничено, то последовательность $\{M_k\}$ ограничена, и поэтому по теореме \hyperref[ch1.1thm3]{Больцано-Вейерштрасса} у нее есть сходящаяся подпоследовательность $\{M_{k_p}\}$. Пусть
$$
\lim\limits_{p\to\infty} M_{k_p} =M_0.
$$
Отсюда и из условия $|M_kM_k'|<1/k$ следует, что 
$$
\lim\limits_{p\to\infty} M_{k_p}' =M_0.
$$
А так как множество $G$ замкнуто, то $M_0 \in G$.

Функция $f(M)$ непрерывна в точке $M_0$, поэтому
$$
|f(M_{k_p})-f(M_{k_p}')|\le |f(M_{k_p})-f(M_0)|+|f(M_0)-f(M_{k_p}')|\to 0
$$
при $p\to\infty$, что противоречит условию \eqref{yaa45e4}, которое следует из нашего предположения. Следовательно, это предположение неверное. Теорема доказана.
\end{proof}

Эту теорему иногда называются \textit{теоремой Кантора о равномерной непрерывности}. Кратко ее формулируют так:

\textit{Функция, непрерывная на компакте, равномерно непрерывна.}
\begin{cons}
Если функция непрерывна на некотором отрезке, то она равномерно непрерывна на этом отрезке. 
\end{cons}

















