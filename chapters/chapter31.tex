\chapter{Математическое ожидание и дисперсия случайной величины, их свойства.}
\section{Математическое ожидание и дисперсия случайной величины, их свойства}

\subsection{Случайные величины}

Пусть дано конечное вероятностное пространство $(\Omega,\mathcal A, P)$, где $P$ --- вероятности каждого из элементарных исходов из $\Omega$. Тогда \textit{случайной величиной} принято называть любую функцию $\xi\colon \Omega \to \bbR$.(т.е. случайному элементарному исходу ставится в соответствие совершенно конкретное значение.)

Абсолютно так же определяется случайная величина для бесконечного счетного вероятностого пространства, где $\Omega = {w_1,\dots, w_n,...}$.

Пусть теперь дано произвольное вероятностное пространство $(\Omega,\mathcal A, P)$, где $P$ --- вероятности каждого из элементарных исходов из $\Omega$. Тогда в этом общем случае:
\begin{defn}
\textit{Случайной величиной} называется действительная функция от элементарного события $\xi = \xi(w), \; w\in\Omega$, для которой при любом действительном $x$ множество $\{w: \xi(w) \le x\}$ принадлежит $\mathcal A$ (т. е. является событием) и для него определена вероятность $P(w: \xi(w) \le x)$, записываемая кратко $F_\xi(x)=P(\xi \le x)$. Эта вероятность, рассматриваемая как функция $x$, называется \textit{функцией распределения случайной величины $\xi$}.
\end{defn}
Отметим ее свойства:
\begin{enumerate}
\item 
$0 \le F(x) \le 1 \fa x;$ 
\item
$F(x_1) \le F(x_2), \text{если}\; x_1\le x_2;$
\item
$F(-\infty)=0, \; F(+\infty)=1;$
\item
$F(x+0)=F(x)$ -- непрерывна слева.
\end{enumerate}
Важным классом распределении вероятностей являются \textit{абсолютно непрерывные распределения}, задаваемые плотностью вероятности $p_\xi(x) = р(х)$, т. о. такой неотрицательной функцией $р(х)$, что для любого события $B\in\Omega$.
$$
P(\xi \in B)=\int_B p(x)\,dx,
$$
Тогда функция распределения $F_\xi(x)=\int_{-\infty}^{x}p(x)\,dx$, где $p(x)$ называют \textit{плотностью вероятности}, обладающая следующими свойствами:
\begin{enumerate}
\item 
$p(x)\le 0$
\item 
$\forall x_1,x_2: P(x_1\le\xi<x_2)=\int_{x_1}^{x_2}p(x)\,dx$
\item
$\int_{-\infty}^{+\infty}p(x)\,dx=1$
\end{enumerate}

Другой класс составляют \textit{дискретные распределения}, задаваемые конечным или счетным набором вероятностей $Р(\xi=x_k)$ для которых
$$
\sum\limits_k P(\xi=x_k)=1,
$$
тогда функция распределения $F_\xi(x)=\sum\limits_{k: x_k \le x} P(\xi=x_k)$
£слн распределение случайной величины абсолютно непрерывно или дискретно, то говорят также, что сама случайная величина или ее функция распределения соответственно абсолютно непрерывны или дискретны.

\section{Математическое ожидание и его свойства}

Сначала дадим определение для дискретных случайных величин. Пусть $x_1,x_2,\dots,x_n,\dots$ обозначают возможные значения случайной величины $\xi$, а $p_1,p_2,\dots,p_n,\dots$ --- соответствующие им вероятности, $\Omega = \{w_1,\dots, w_n,...\}$ - наше пространство элементарных исходов в $(\Omega,\mathcal A, P)$.
\begin{defn}\label{teorver1}
\textit{Математическим ожиданием случайной величины} $\xi$ называется число, обозначаемое $M\xi$ и равное
\begin{equation}
M\xi = \sum_{w \in \Omega} \xi(w)\cdot P(w),
\end{equation}
где $P(w)$ --- элементарные вероятности.
\end{defn}
Перепишем определение матожидания по-другому.
\begin{multline*}
M\xi=\sum_{w \in \Omega} \xi(w)\cdot P(w)=x_1\left(\textstyle\sum\limits_{w\colon \xi(w)=x_1} P(w)\right)+\dots+x_k\left(\textstyle\sum\limits_{w\colon \xi(w)=x_k} P(w)\right)+\dots=\\=x_1P(\xi=x_1)+\dots x_kP(\xi=x_k)+\dots=\sum_{k\in \bbN} x_kP(\xi=x_k)
\end{multline*}
В силу этого равентсва, дадим аналогичное определение матожиданию:
\begin{defnn}{\ref{teorver1}'}\label{teorver1s}
Если ряд $\sum\limits_{k=1}^{\infty}x_k p_k$ сходится абсолютно, то его сумма называется \textit{математическим ожиданием случайной величины} $\xi$.
\end{defnn}


Для непрерывных случайных величин естественным будет следующее определение: 
\begin{defn} 
Если случайная величина $\xi$ непрерывна и $р(x)$ --- ее плотность распределения, то \textit{математическим ожиданием} \textit{случайной величины} $\xi$ называется интеграл Для произвольной случайной величины $\xi$ с функцией распределения $F(x)$ математическим ожиданием называется интеграл
\begin{equation}
M\xi\triangleq\int_{-\infty}^{+\infty} xp(x)\,dx = \int_{-\infty}^{+\infty} x\,dF_\xi(x),
\end{equation}
в тех случаях, когда существует интеграл $\int |x|p(x)\,dx.$ 
\end{defn}
\begin{thm}\label{thm1}
Математическое ожидание постоянной равно этой постоянной.
\end{thm}
\begin{proof}
Постоянную $C$ мы можем рассматривать, как дискретную случайную величину, которая может принимать только одно значение $C$ с вероятностью единица; поэтому
$$
\ccM C = C\cdot1=C
$$
\end{proof}
\begin{thm}[линейность] Для любых случайных величин $\xi_1$ и $\xi_2$ и любых чисел $c_1$ и $c_2$ справедливо
\begin{equation}
\ccM(c_1\xi_1+c_2\xi_2) = с_1\ccM\xi_1+c_2\ccM\xi_2.
\end{equation}
\end{thm}

\section{Дисперсия и ее свойства}
\begin{defn}
\textit{Дисперсией случайной величины $\xi$} называется математическое ожидание квадрата уклонения $\xi$ от $M\xi$. Обозначим ее $\ccD\xi$. 
\begin{equation}
\ccD\xi=\ccM(\xi-\ccM\xi)^2.
\end{equation}
\end{defn}

Дисперсия играет роль меры рассеяния(разбросанности) значений случайной величины около математического ожидания.

Заметим, что в силу линейности мат.ожидания и теоремы \ref{thm1}(Поскольку матожидание это постоянное число, то $\ccM(\ccM\xi)=\ccM\xi$:
\begin{multline*}
\ccD\xi = \ccM(\xi-\ccM\xi)^2=\ccM(\xi^2-2\xi \ccM\xi+(\ccM\xi)^2)=\ccM\xi^2-2\ccM\xi \ccM\xi+(\ccM\xi)^2=\ccM\xi^2-(\ccM\xi)^2
\end{multline*}
Так как дисперсия является неотрицательной величиной, то из последнего мы выводим еще одно свойство матожидания: $\ccM\xi^2\ge(\ccM\xi)^2$.
















