\chapter{Неравенство Чебышёва и закон больших чисел. Предельная теорема Пуассона.}
\section{Неравенство Чебышёва и закон больших чисел}
\begin{thm}[неравенство Маркова]
Пусть случайная величина $\xi$ принимает неотрицательные значения. Пусть $a>0$. Тогда 
$$
P(\xi \ge a)\le\frac{\ccM\xi}{a}.
$$
\end{thm}
\begin{proof}
Представим $\xi$ в виде суммы двух неотрицательных случайных величин
$$
\xi = \xi\cdot\chi_{\{\xi\ge a\}}+\xi\cdot\chi_{\{\xi< a\}}.
$$
По свойству аддитивности, монотонности имеем:
$$
\ccM\xi=\ccM(\xi\cdot\chi_{\{\xi\ge a\}})+\underbrace{\ccM(\xi\cdot\chi_{\{\xi< a\}})}_{\ge 0} \ge \ccM(\xi\cdot\chi_{\{\xi\ge a\}}).
$$
Так как $\xi\cdot\chi_{\{\xi\ge a\}}\ge a\cdot\chi_{\{\xi\ge a\}}$, то применяя еще раз свойство монотонности, получаем
$$
\ccM\xi\ge a\ccM(\xi\cdot\chi_{\{\xi\ge a\}})=a\cdot P(\xi\ge a),
$$
что и доказывает неравенство Маркова.
\end{proof}

\begin{thm}[неравенство Чебышёва]
Если случайная величина имеет дисперсию $\ccD\xi$, и пусть $b>0$, то
$$
P(|\xi-\ccM\xi|\ge b ) \le\frac{\ccD\xi}{b^2}.
$$
\end{thm} 
\begin{proof}
Уже доказано неравенство Маркова $P(\eta \ge a) \le \frac{M\eta}{a}$  для $\eta \ge 0$, $a>0$. Полагая в нем $\eta = |\xi-\ccM\xi|^2$ и $a=b^2$, получаем неравенство Чебышёва.
\end{proof}



\section{Предельная теорема Пуассона}