\chapter{Исследование функций одной переменной при помощи первой и второй производных на монотонность, локальные экстремумы, выпуклость. Необходимые условия, достаточные условия.}

\section{Исследование функций одной переменной при помощи первой и второй производных на монотонность, локальные экстремумы, выпуклость; необходимые условия, достаточные условия}

\subsection{Определение производной}
Пусть задана функция $f$ и точка $x_0\in D_f$. Тогда для любого $x \in D_f$, $x\ne x_0$, частное
\begin{equation}\label{ya3e1}
\frac{f(x)-f(x_0)}{x-x_0}=\frac{f(x_0+h)-f(x_0)}{h}
\end{equation}
называется \textit{разностным отношением функции $f$ в точке $x_0$ с шагом $h$}.

\begin{defn}
\label{ya3d1}
Предел разностного отношения функции $f$ в точке $x_0 \in D_fS$ с шагом $h$ при $h \to 0$ называется \textit{производной функции} $f$ в точке $x_0$ и обозначается $f'(x_0)$

\begin{equation}\label{ya3e2}
f'(x_0)=\lim_{h\to 0}\limits\frac{f(x_0+h)-f(x_0)}{h}=\lim_{x\to x_0}\limits \frac{f(x)-f(x_0)}{x-x_0}
\end{equation}

\end{defn}
\begin{defn} Предел разностного отношения функции $f$ в точке $x_0 \in D_f$ с шагом $h$ при $h \to +0 $ ($h \to -0$) называется правой (левой) производной функции / в точке Хо и обозначается /+(хо) (соотв., /1(х0)).
\begin{equation}\label{ya3e3}
f'_\pm(x_0)=\lim_{h\to \pm0}\limits\frac{f(x_0+h)-f(x_0)}{h}
\end{equation}
\end{defn}

Очевидно, функция, определенная в некоторой окрестности точки $x_0$, имеет производную в точке $x_0$ тогда и только тогда, когда она в $x_0$ имеет односторонние производные и эти производные равны.


Заметим, что пределы (2) и (3) могут быть как конечными, гак и бесконечными, и поэтому можно говорить о конечных и бесконечных производных. \textbf{В дальнейшем выражение «функция имеет производную» означает, что функция имеет конечную производную. Случай бесконечных производных оговаривается особо. !!! Проверить, употребляю я ли я вообще выражение функция имеетпроизводную без слово конечная}

\begin{defn} Функция $f$ называется \textit{дифференцируемой в точке} $x_0$, если она определена в некоторой окрестности точки $x_0$ и в этой точке имеет конечную производную.
\end{defn}
Формулы \eqref{ya3e2} и \eqref{ya3e3} часто записывают в других обозначениях. Вместо $x_0$ пишут $x$, шаг $h$ разностного отношения \eqref{ya3e1} обозначают $\Delta x$, и тогда	
\begin{equation}\label{ya3e4}
f'(x)=\lim_{\Delta x\to 0}\limits\frac{f(x+\Delta)-f(x)}{\Delta x}
\end{equation}
\textbf{KAK TO SOKRATIT' SLED ABZAC}
Здесь $\Delta x$ называется приращением аргумента, а разность $f(x+\Delta)-f(x)$ --- соответствующим приращением функции. Если $y = f(x)$, то это приращение функции обозначают $\Delta y$, а производную функции $f$ в точке $x$ обозначают $y'$ В этих обозначениях формула \eqref{ya3e4} принимает вид
$$
y'=\lim_{\Delta x \to 0}\limits \frac{\Delta y}{\Delta x}
$$
Аналогичным образом записываются и формулы \eqref{ya3e3}. В такой форме определение производной коротко формулируют так:
Производной называется предел отношения приращения функции к приращению аргумента, когда приращение аргумента стремится к нулю.	

\begin{defn}
Пусть $D'_f$ --- множество точек, в которых функция $f$ имеет конечную производную. Тогда функция, которая каждому $x \in D'_f$ ставит в соответсвие число $f'(x)$, называется \textit{производной} функции $y=f(x)$ и обозначается $f'$ иди $y'$
\end{defn}
Операция нахождения производной данной функции $f$ называется \textit{дифференцированием функции} $f$.


\begin{defn}
Производня производной $f'$ функции $f$ в точке $x_0 \in D'_f$ называется \textit{второй производной}(или \textit{производной второго порядка}) \textit{функции $f$ в точке $x_0$} и обозначается $f''(x_0)$ или $f^(2)(x_0)$.
\end{defn}
Вообще, производная производной $f'$ функции $f$ называется \textit{второй производной} функции $y=f(x)$ и обозначается $f''$, $f^(2)$ или $y''$, $y^(2)$


\subsection{Исследование функций одной переменной при помощи первой и второй производных на монотонность, локальные экстремумы, выпуклость; необходимые условия, достаточные условия}

ПОЛНЫЙ БРЕД. 