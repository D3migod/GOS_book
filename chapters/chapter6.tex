\chapter{Исследование функций одной переменной при помощи первой и второй производных на монотонность, локальные экстремумы, выпуклость. Необходимые условия, достаточные условия.}

\section{Определение производной}
Пусть задана функция $f$ и точка $x_0\in D_f$. Тогда для любого $x \in D_f$, $x\ne x_0$, частное
\begin{equation}\label{ya3e1}
\frac{f(x)-f(x_0)}{x-x_0}=\frac{f(x_0+h)-f(x_0)}{h}
\end{equation}
называется \textit{разностным отношением функции $f$ в точке $x_0$ с шагом $h$}.

\begin{defn}
\label{ya3d1}
Предел разностного отношения функции $f$ в точке $x_0 \in D_fS$ с шагом $h$ при $h \to 0$ называется \textit{производной функции} $f$ в точке $x_0$ и обозначается $f'(x_0)$

\begin{equation}\label{ya3e2}
f'(x_0)=\lim_{h\to 0}\limits\frac{f(x_0+h)-f(x_0)}{h}=\lim_{x\to x_0}\limits \frac{f(x)-f(x_0)}{x-x_0}
\end{equation}

\end{defn}
\begin{defn} Предел разностного отношения функции $f$ в точке $x_0 \in D_f$ с шагом $h$ при $h \to +0 $ ($h \to -0$) называется правой (левой) производной функции $f$ в точке $x_0$ и обозначается $f_+(x_0)$ (соотв., $f_-(x_0)$).
\begin{equation}\label{ya3e3}
f'_\pm(x_0)=\lim_{h\to \pm0}\limits\frac{f(x_0+h)-f(x_0)}{h}
\end{equation}
\end{defn}

Очевидно, функция, определенная в некоторой окрестности точки $x_0$, имеет производную в точке $x_0$ тогда и только тогда, когда она в $x_0$ имеет односторонние производные и эти производные равны.


Заметим, что пределы (2) и (3) могут быть как конечными, гак и бесконечными, и поэтому можно говорить о конечных и бесконечных производных. \textbf{В дальнейшем выражение «функция имеет производную» означает, что функция имеет конечную производную. Случай бесконечных производных оговаривается особо. !!! Проверить, употребляю я ли я вообще выражение функция имеетпроизводную без слово конечная}

\begin{defn} Функция $f$ называется \textit{дифференцируемой в точке} $x_0$, если она определена в некоторой окрестности точки $x_0$ и в этой точке имеет конечную производную.
\end{defn}
Формулы \eqref{ya3e2} и \eqref{ya3e3} часто записывают в других обозначениях. Вместо $x_0$ пишут $x$, шаг $h$ разностного отношения \eqref{ya3e1} обозначают $\Delta x$, и тогда	
\begin{equation}\label{ya3e4}
f'(x)=\lim_{\Delta x\to 0}\limits\frac{f(x+\Delta)-f(x)}{\Delta x}
\end{equation}
\textbf{KAK TO SOKRATIT' SLED ABZAC}
Здесь $\Delta x$ называется приращением аргумента, а разность $f(x+\Delta)-f(x)$ --- соответствующим приращением функции. Если $y = f(x)$, то это приращение функции обозначают $\Delta y$, а производную функции $f$ в точке $x$ обозначают $y'$ В этих обозначениях формула \eqref{ya3e4} принимает вид
$$
y'=\lim_{\Delta x \to 0}\limits \frac{\Delta y}{\Delta x}
$$
Аналогичным образом записываются и формулы \eqref{ya3e3}. В такой форме определение производной коротко формулируют так:
Производной называется предел отношения приращения функции к приращению аргумента, когда приращение аргумента стремится к нулю.	

\begin{defn}
Пусть $D'_f$ --- множество точек, в которых функция $f$ имеет конечную производную. Тогда функция, которая каждому $x \in D'_f$ ставит в соответсвие число $f'(x)$, называется \textit{производной} функции $y=f(x)$ и обозначается $f'$ иди $y'$
\end{defn}
Операция нахождения производной данной функции $f$ называется \textit{дифференцированием функции} $f$.


\begin{defn}
Производня производной $f'$ функции $f$ в точке $x_0 \in D'_f$ называется \textit{второй производной}(или \textit{производной второго порядка}) \textit{функции $f$ в точке $x_0$} и обозначается $f''(x_0)$ или $f^(2)(x_0)$.
\end{defn}
Вообще, производная производной $f'$ функции $f$ называется \textit{второй производной} функции $y=f(x)$ и обозначается $f''$, $f^{(2)}$ или $y''$, $y^{(2)}$


\section{Исследование функций одной переменной при помощи первой и второй производных на монотонность}

Имеет место следующее следствие из \hyperref[ch4t1]{теоремы Лагранжа о среднем}
\begin{thm} (критерий постоянства дифференцируемой функции)
Если функция $f(x)$ дифференцируема на интервале , то $f(x)$ постоянна на $(a;b)$ тогда и только тогда, когда $f'(x)=0$ на $(a;b)$.
\end{thm}
 
\begin{thm} (критерий возрастания/убывания дифференцируемой функции.) Если функция $f(x)$ дифференцируема на интервале $(a;b)$, то $f(x)$ возрастает (убывает) на $(a;b)$ тогда и только тогда, когда $f'(x)>0$ (соотв., $f'(x)<0$) на $(a;b)$.
\end{thm}
\begin{proof} Если $f(x)$ возрастает на $(a;b)$, то для любых $x$ и $x+\Delta x$ из $(a;b)$ разность $\Delta f = f(x+\Delta x) - f(x)$ имеет тот же знак, что и $\Delta x$, и поэтому всегда $\Delta f/\Delta x > 0$. Отсюда в пределе при $\Delta x \to 0$ следует, что $f'(x) > 0$. Аналогично, если $f(x)$ убывает на $(a;b)$, то $\Delta f/\Delta x < 0$, и поэтому $f'(x) < 0$ на $(a;b)$.

Наоборот, для любых $x$ и $x + \Delta x$ из $(a;b)$, по теореме Лагранжа, существует $\xi$ такое, что $\Delta f = f(\xi)\Delta x$. Поэтому, если $f'(x) \ge 0$ ($\le 0$) на $(a;b)$, то $f(x)$ возрастает (убывает) на $(a;b)$.

\noindent Теорема доказана.
\end{proof}

\begin{thm} (достаточное условие строго возрастания/убывания дифференцируемой функции) Если функция $f(x)$ на интервале $(a;b)$ дифференцируема и $f'(x) > 0$ ($< 0$) на $(a;b)$, то она строго возрастает (убывает) на $(a;b)$.
\end{thm}
\begin{proof}
По теореме Лагранжа, для любых $x$ и $x + \Delta x$ из $(a; b)$ существует $\xi$ такое, что
$$
f(x+\Delta x)-f(x)=f'(\xi)\Delta x.
$$
Отсюда следует, что если $f'(x) > 0$, то $f(x)$ строго возрастает, а если $f'(x) < 0$, то $f(x)$ строго убывает на $(a;b)$. 
\noindent
Теорема доказана.
\end{proof}
Заметим, что условие: $f'(x) > 0$ на $(a;b)$, являясь достаточным, не является необходимым для строго возрастания дифференцируемой функции $f(x)$ на $(a;b)$. Например, функция $f(x) = x^3$ строго возрастает на $\bbR$, но $f'(0) = 0$.

\section{Исследование функций одной переменной при помощи первой и второй производных на локальные экстремумы}
Точки экстремума и экстремальные значения функции (локальные минимумы и локальные максимумы) определялись в Билете 4. Там же была доказана теорема Ферма:

\begin{thm}(необходимое условие экстремума дифференцируемой функции)
Если функция $f(x)$ в точке $x_0$ дифференцируема и $x_0$ является точкой экстремума для $f(x)$, то $f'(x_0) = 0$.
\end{thm}

Заметим, что условие $f'(x_0) = 0$, являясь необходимым, не является достаточным для того, чтобы точка $x_0$ была экстремальной для $f(x)$. Например, для$ f(x) = x^3$ имеем: $f'(0) = 0$; но точка $x_0 = 0$ не является экстремальной.

\begin{defn} Точка $x_0$ называется \textit{стационарной точкой функции} $f(x)$, если $f(x)$ в точке $x_0$ дифференцируема и $f'(x_0) = 0$. Если же $f'(x_0) > 0$ ($< 0$), то $x_0$ называется точкой возрастании (убывания) функции $f(x)$.
\end{defn}

Из теоремы Ферма следует, что точки экстремумов данной функции следует искать среди так называемых, \textit{критических точек}, т.е. среди стационарных точек и точек, в которых нет производной.

\begin{defn}Точка $x_0 \in D_f$ называется \textit{точкой строгого максимума (минимума) функции} $f(x)$, если существует окрестность $\dnei{x_0}$ точки $x_0$ такая, что
$$
\forall x \in \dnei{x_0}\cap D_f \quad f(x) < f(x_0) \quad(\text{соотв.} f(x) > f(x_0)).
$$
\end{defn}

Точки строгого максимума и минимума функции называют \textit{точками строгого экстремума} этой функции.

\begin{thm}(достаточное условие) Пусть функция $f(x)$ непрерывно на интервале $(a;b)$ и дифференцируема на интервалах $(a;x_0)$ и $(x_0;b)$. Тогда, если $f'(x) > 0$ на $(a;x_0)$ и $f'(x) < 0$ на $(x_0;b)$, то $x_0$ --- точка строгого максимума, а если $f'(x) < 0$ на $(a;x_0)$ и $f'(x) > 0$ на $(x_0;b)$, то $x_0$ --- точка строгого минимума функции $f(x)$.
\end{thm}
\begin{proof}
Пусть выполнено $f'(x) > 0$ на $(a;x_0)$ и $f'(x) < 0$ на $(x_0;b)$. Пусть $x \in (a;x_0) \Rightarrow$ на $[x;x_0]$ выполняются все условия теоремы Лагранжа. $\Rightarrow$ 
$$
\exists \xi \in (x;x_0):\quad f(x_0)-f(x) = \underbrace{f'(\xi)}_{>0} \underbrace{(x_0-x)}_{>0} >0 \Rightarrow \forall x \in (a;x_0)\quad f(x_0) < f(x).
$$
Аналогично, пусть $x\in(x_0;b)$. Тогда
$$
\exists \xi \in (x_0;x):\quad f(x_0)-f(x) = \underbrace{f'(\xi)}_{<0} \underbrace{(x_0-x)}_{<0} >0 \Rightarrow \forall x \in (x_0;b) \quad f(x_0) < f(x).
$$
Т.е. $x_0$ --- точка строго максимума функции $f$.

\noindent
Теорема доказана.
\end{proof}
Доказанную теорему образно формулируют следующим образом. \textit{Если при переходе через точку $x_0$ производная $f'(x)$ меняет знак с плюса на минус, то $x_0$ --- точка максимума, а если —-- с минуса на плюс, то $x_0$ --- точка минимума функции $f(x)$.}
\begin{thm} (достаточное условие) Пусть функция $f(x)$ в точке $x_0$ имеет конечную производную 2-го порядка. Тогда, если $f'(xо) = 0$, а $f(x_0) > 0$ ($<0$), то $x_0$ --- точка строгого минимиума (максимума) функции f(x).
\end{thm}

\begin{proof}
По формуле Тейлора с остаточным членом в форме Пеано имеем:	
$$
f(x)=f(x_0)+\left(\frac{1}{2}f''(x_0)+\alpha(x)\right)(x-x_0)^2
$$
где $\alpha \to 0$ при $x \to x_0$.
 
Если $f''(x_0) > 0$, то $\exists O(x_0) \fa x\in\dnei{x_0}\quad\frac{1}{2}f''(x_0)+\alpha(x)>0$
и поэтому $f(x) > f(x_0) \fa \dnei{x_0}$. Следовательно, $x_0$ --- точка строгого минимума функции $f(x)$.
Аналогично доказывается, что если $ f''(x_0) < 0$, то $x_0$ — точка строгого максимума функции $f(x)$.

\noindent
Теорема доказана.
\end{proof}


\section{Исследование функций одной переменной при помощи первой и второй производных на выпуклость}
